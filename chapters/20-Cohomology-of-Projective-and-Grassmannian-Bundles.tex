\chapter[Cohomology of Projective and Grassmannian Bundles]{Cohomology of Projective and\\ Grassmannian Bundles}
In this chapter we calculate the cohomology of the total space of certain smooth fiber bundles, 
associated to vector bundles, as a module over the cohomology of the base manifold. 
As corollaries we obtain the splitting principles for complex and oriented vector 
bundles used in Chapters 18 and 19.

Let $\pi:E\ra M$ be a smooth fiber bundle over $M$ with fiber $F$. There is a product
\[
  H^{i}( M)  \otimes  H^{j}( E)  \ra  H^{i + j}( E)
\]
given by the formula
\begin{align}\label{eq:20-1}
  a.e = {\pi }^*( a)  \land  e\;\text{ for }a \in  H^{i}( M),e \in  H^{j}( E).
\end{align}

Thus $H^*( E)$ becomes a (graded) module over the (graded) algebra $H^*( M)$. We shall examine 
this module structure in the particular simple case where we suppose given classes ${e}_{\alpha } \in  H^{{n}_{\alpha }}( E)$ 
for $\alpha \in  A$ with the property that for every $p \in  M$,
\begin{align}\label{eq:20-2}
  \left\{  {{i}_{p}^*( {e}_{\alpha })  \mid  \alpha  \in  A}\right\}  
  \quad
  \text{ is a basis for the vector space }H^*( {F}_{p}).
\end{align}

Here ${F}_{p} = {\pi }^{-1}( p)$ is the fiber over $p$ and ${i}_{p}$ is the inclusion of ${F}_{p}$ into $E$.

\begin{theorem}\label{theorem:20-1}
  In the above situation $H^*( E)$ is a free $H^*( M)$-module with basis $\left\{  {{e}_{\alpha } \mid  \alpha  \in  A}\right\}$.  
\end{theorem}

\begin{proof}
  The proof follows the pattern used to prove Poincaré duality in Chapter 13. Let $\mathcal{V}$ be the cover 
  consisting of open sets $V \subset  M$, such that $E$ is trivial over $V$. Let $\mathcal{U}$ be the 
  cover of $M$ by open sets, so that the theorem is satisfied with $M$ replaced by $U \in  \mathcal{U}$ and $E$ 
  replaced by ${\pi }^*( U)$. We must verify the conditions of Theorem \ref{theorem:13-9}. We leave conditions (i), (ii) 
  and (iv) to the reader and prove condition (iii). So suppose
  \[
    U = {U}_{1} \cup  {U}_{2},\;{U}_{12} = {U}_{1} \cap  {U}_{2}
  \]
  and let ${E}_{1},{E}_{2}$ and $E_{12}$ denote the restriction of the bundle $E$ over ${U}_{1},{U}_{2}$ 
  and ${U}_{12}$. The classes ${e}_{\alpha } \in  H^{{n}_{\alpha }}( E)$ restrict to classes which again 
  satisfy condition \eqref{eq:20-2}, and we denote the restricted classes by the same letters. We suppose that the theorem 
  is true for $H^*( {E}_{1}),H^*( {E}_{2}),H^*( E_{12})$, and want to conclude it is 
  true for $H^*( {E}_{U})$. This employs the two Mayer-Vietoris sequences
  \begin{align*}
    & \cdots \overset{{J}^*}
      { \ra  }H^{*- 1}( E_{12}) \overset{{\delta }^*}
      { \ra  }H^*( E) \overset{{I}^*}
      { \ra  }H^*( {E}_{1})  \oplus  H^*( {E}_{2}) \overset{{J}^*}
      { \ra  }\cdots\\
    & \cdots \overset{{J}^*}
      { \ra  }H^{*- 1}( {U}_{12}) \overset{{\delta }^*}
      { \ra  }H^*( U) \overset{{I}^*}
      { \ra  }H^*( {U}_{1})  \oplus  H^*( {U}_{2}) \overset{{J}^*}
      { \ra  }\cdots
  \end{align*}
  where we write $E$ instead of ${E}_{U}$. We must show that every element $e \in  H^*( E)$ has a unique 
  representation of the form $e = \sum {m}_{\alpha }{e}_{\alpha }$ with ${m}_{\alpha } \in  H^*( U)$. 
  We give the existence proof and leave uniqueness to the reader. By assumption we know that
  \[
    {i}_{\nu }^*( e)  = \sum {m}_{\alpha }^{( \nu ) } \cdot  {e}_{\alpha },\quad\nu  = 1,2
  \]
  where ${i}_{\nu } : {E}_{\nu } \ra  E$ is the inclusion. Since ${J}^*{I}^* = 0$,
  \[
    \sum {j}_{1}^*( {m}_{\alpha }^{( 1) }) {e}_{\alpha } 
    = \sum {j}_{2}^*( {m}_{\alpha }^{( 2) }) {e}_{\alpha }
  \]
  in $H^*( E_{12})$, where ${j}_{\nu } : E_{12} \ra  {E}_{\nu }$ is the inclusion.

  Uniqueness of representations for $H^*(E_{12})$ shows that ${j}_{1}^*( {m}_{\alpha }^{( 1) })= {j}_{2}^*( {m}_{\alpha }^{( 2) })$ 
  for each $\alpha  \in  A$, and the Mayer-Vietoris sequence for the base spaces implies elements ${m}_{\alpha } \in  H^*( U)$ 
  with ${I}^*( {m}_{\alpha })  = ( {{m}_{\alpha }^{( 1) },{m}_{\alpha }^{( 2) }})$, so that
  \[
    {I}^*( {e- \sum {m}_{\alpha }{e}_{\alpha }})  = 0.
  \]

  It thus suffices to argue that every element of $\R{Ker}{I}^* = \R{Im}{\delta }^*$ has 
  a representation as asserted. This in turn is an easy consequence of the theorem for $H^*( E_{12})$ 
  and the formula
  \begin{align}\label{eq:20-3}
    {\delta }^*( {m \cdot  {i}_{12}^*( e) })  = {\delta }^*( m)  \cdot  e,
  \end{align}
  valid for any $m \in  H^*( U)$ and $e \in  H^*( E)$, with ${i}_{12} : E_{12} \ra  E$ the inclusion. 
  We leave the proof of \eqref{eq:20-3} as an exercise.
\end{proof}

We are now ready to prove the complex splitting principle, as stated in Theorem \ref{theorem:18-10}. 
Let $\xi$ be a complex vector bundle over $M$ with ${\dim }_{\B{C}}\xi  = n + 1$. We form an associated 
fiber bundle $P( \xi )$ over $M$ with total fiber space
\[
  E( {P( \xi ) })  = \left\{  {( {p,L})  \mid  p \in  M,L \in  P( {\xi_p}) }\right\} .
\]

Here $P( {\xi_p})$ denotes the projective space of complex lines in the vector space ${\xi_p}$. Projection 
onto the first factor
\[
  \pi: E( {P( \xi ) })  \ra  M
\]
makes $P( \xi )$ into a fiber bundle over $M$. We leave the reader to show that $P( \xi )$ is a smooth manifold 
and that $\pi$ is a proper smooth map. There is a complex line bundle $H( \xi )$ over $P( \xi )$ with total space
\[
  E( {H( \xi ) })  = \{ ( {p,L,v})  \mid  ( {p,L})  \in  P( \xi ),v \in  L\}.
\]

If $M$ consists of a single point then $P( \xi )$ is the complex projective $n$-space ${\CP^n}$ and $H( \xi )$ 
is the canonical line bundle of Example \ref{example:15-2}. If more generally $\xi  = M \times  {\B{C}}^{n + 1}$ is the trivial 
bundle then $P( \xi )  = M \times  {\CP^n}$ and $H( \xi )  = {\R{pr}}_{2}^*( H_{n})$. Let us give $\xi$ 
an inner product. Then ${\pi }^*( \xi )$ has an inner product, and we can form the fiberwise orthogonal 
complement $H{( \xi ) }^{ \bot  }$ of the subbundle $H( \xi )  \subset  {\pi }^*( \xi )$, i.e.
\[
  H{( \xi ) }^{ \bot  } = \left\{  {( {p,L,v})  \mid  ( {p,L})  \in  P( \xi ),v \in  {L}^{ \bot  }}\right\}
\]
where the orthogonal complement ${L}^{ \bot  }$ is calculated in the fiber ${\xi_p}$. 
Clearly $L \oplus  {L}^{ \bot  } = {\xi_p} = {\pi }^*{( \xi ) }_{p}$. So that we have 
an isomorphism of vector bundles
\begin{align}\label{eq:20-4}
  {\pi }^*( \xi )  = H( \xi )  \oplus  H{( \xi ) }^{ \bot  }
\end{align}

Let $e$ be the first Chern class of $H( \xi ),e = {c}_{1}( {H( \xi ) })$. We want to apply Theorem \ref{theorem:20-1} 
to the classes
\begin{align}\label{eq:20-5}
  1,e,{e}^{2},\ldots,{e}^{n} \in  H^*( {P( \xi ) }).
\end{align}

Property \eqref{eq:20-2} is satisfied because the fiber of $\pi: P( \xi )  \ra  M$ over $p \in  M$ is the projective 
space $P( {\xi_p})  = {\CP^n}$, and because the restriction of $H( \xi )$ to $P( {\xi_p})$ is the canonical 
line bundle $H_{n}$ over ${\CP^n}$. Now ${i}_{p}^*( e)  = {c}_{1}( H_{n})  \neq  0$ and the powers ${e}^{i}$ 
restrict by ${i}_{p}^*$ to ${c}_{1}{( H_{n}) }^{i}$ which are non-zero in $H^{2i}( {\CP^n})$, and hence a basis
as long as $i \leq  n$, by Theorem \ref{theorem:14-3}.

In the situation of Theorem \ref{theorem:20-1} one has in particular that ${\pi }^* : H^*( M)  \ra  H^*( E)$ is injective. 
Indeed ${\pi }^*( m)  = {m.1}$, and 1 is an $\B{R}$-linear combination of basis elements ${e}_{\alpha }$. 
We have proved:

\begin{theorem}\label{theorem:20-2}
  For any complex $n$-dimensional vector bundle $\xi$ over $M,H^*( {P( \xi ) })$ is a free $H^*(M)$-module with basis
  \[
    1,{c}_{1}( {H( \xi ) }),\ldots,{c}_{1}{( H( \xi ) ) }^{n- 1}.
  \]
  In particular, ${\pi }^* : H^*( M)  \ra  H^*( {P( \xi ) })$ is injective.
\end{theorem}

We may now prove the splitting principle for complex vector bundles.

\begingroup
\def\proofname{\textup{\bfseries Proof of Theorem \ref{theorem:18-10}}} 
\begin{proof}
  Starting with $\xi$ over $M$ with ${\dim }_{\mathrm{C}}\xi  = n + 1$, we consider the composition
  \[
    P( {\xi }_{n- 1}) \overset{{\pi }_{n- 1}}
    { \ra } \cdots  
    { \ra }  P( {\xi }_{1}) \overset{{\pi }_{1}}
    { \ra  }P( \xi ) \overset{{\pi }_{0}}
    { \ra  }M
  \]
  where ${\xi }_{1} = H{( \xi ) }^{ \bot  }$ was defined above and where ${\xi }_{2},H( {\xi }_{1})$ are 
  the corresponding bundles over $P( {\xi }_{1})$, i.e. ${\xi }_{2} \oplus  H( {\xi }_{1})  = {\pi }_{1}^*( {\xi }_{1})$ 
  etc. Thus if we let $f = {\pi }_{0} \circ  \ldots  \circ  {\pi }_{n- 1}$, ${f}^*( \xi )$ is the sum of 
  the pull-backs of the line bundles $H( {\xi }_{i})$ over $P( {\xi }_{i})$, and ${f}^* = {\pi }_{n- 1}^* \circ  \ldots  \circ  {\pi }_{0}^*$ 
  is injective by Theorem \ref{theorem:20-2}.
\end{proof}
\endgroup

The above discussion contains no statement about the class ${e}^{n + 1} = {c}_{1}{( H( \xi ) ) }^{n + 1}$ in $H^{{2n} + 2}( {P( \xi ) })$ 
except of course that
\[
  {c}_{1}{( H( \xi ) ) }^{n + 1} 
  = {\lambda }_{0}( \xi ) {\cdot 1} 
    + {\lambda }_{1}( \xi )\cdot e 
    + \cdots  
    + {\lambda }_{n}( \xi )\cdot {e}^{n}
\]
for some uniquely determined classes
\[
  {\lambda }_{i}( \xi )  \in  H^{{2n} + 2- {2i}}( M).
\]

We assert that
\begin{align}\label{eq:20-6}
  {\lambda }_{i}( \xi )  = {(-1) }^{i + 1}{c}_{n + 1- i}( \xi ).
\end{align}

To see this we use that ${\pi }_{1}^*( \xi )  = H( \xi )  \oplus  {\xi }_{1}$ and the exponential 
property of the total Chern class so that $c( {H( \xi ) }) c( {\xi }_{1})  = c( {{\pi }_{1}^*( \xi ) })$. 
Hence
\[
  c( {\xi }_{1})  
  = {\pi }^*( {c( \xi ) })  \land  c{( H( \xi ) ) }^{-1} 
  = c( \xi )  \cdot  {( 1 + {c}_{1}( H( \xi ) ) ) }^{-1}.
\]

In $H^{{2n} + 2}( {P( \xi ) })$ we get the formula
\[
  {c}_{n + 1}( {\xi }_{1})  
  = \mathop{\sum }\limits_{{i = 0}}^{{n + 1}}{(-1) }^{i}{c}_{n + 1- i}( \xi )  \cdot  {c}_{1}{( H( \xi ) ) }^{i}
\]
which is equivalent to \eqref{eq:20-6}, because ${\dim }_{\B{C}}{\xi }_{1} = n$ and thus ${c}_{n + 1}( {\xi }_{1})  = 0$.

\begin{remark}\label{remark:20-3}
  One can turn the above argument upside down and use \eqref{eq:20-6} to define the Chern classes, 
  once ${c}_{1}( L)$ is defined for a line bundle. One then must show that the Chern classes 
  so defined satisfy the two last conditions of Theorem \ref{theorem:18-10}. This treatment of Chern classes is 
  due to A. Grothendieck. It is useful in numerous situations and gives for example Chern classes 
  in singular cohomology, in $K$-theory and in \'etale cohomology.  
\end{remark}

The rest of this chapter is about the splitting principle for oriented real vector bundles. The construction 
is similar in spirit to the case of complex bundles, but the details are somewhat harder. The projectivized 
bundle $P( \xi )$ is replaced by the bundle $\widetilde{G}_{2}( \zeta )$ of fiberwise oriented 2-planes 
in the oriented real vector bundle $\zeta$ over $M$, and the canonical line bundle $H( \xi )$ over $P( \xi )$ 
is replaced by the oriented 2-plane bundle ${\gamma }_{2} = {\gamma }_{2}( \zeta )$ over $\widetilde{G}_{2}( \zeta )$ 
whose fiber over an oriented plane in ${\zeta }_{p}$ is that plane itself. If $\zeta$ has an inner product 
then ${\pi }^*( \zeta )  = {\gamma }_{2}( \zeta )  \oplus  {\gamma }_{2}{( \zeta ) }^{ \bot  }$ as oriented 
bundles, so that the procedure may be iterated. The analogue of \eqref{eq:20-5} is a set of classes 
in $H^*( {\widetilde{G}_{2}( \zeta ) })$, namely the classes
\begin{align}\label{eq:20-7}
  1, e( {\gamma }_{2}), e{( {\gamma }_{2}) }^{2}, 
  \ldots, e{( {\gamma }_{2}) }^{{2n}- 2}, e( {\gamma }_{2}^{ \bot  })  
  \in  H^*( {\widetilde{G}_{2}( \zeta ) })
\end{align}
where ${\gamma }_{2} = {\gamma }_{2}( \zeta ),{\dim }_{\RR}\zeta  = {2n}$ and where $e(-)$ is the 
Euler class of the previous chapter. In order to apply Theorem \ref{theorem:20-1} we must show that the classes in \eqref{eq:20-7} 
are a basis for $H^*( {\widetilde{G}_{2}( {\B{R}}^{2n}) })$. We now give the details, starting with a proper 
definition of $\widetilde{G}_{2}( {\B{R}}^{2n})$ and then proceeding with the somewhat cumbersome calculation 
of its cohomology.

Let ${V}_{2}( {\RR^m})$ denote the set of orthonormal pairs $(x, y)$ of vectors in ${\RR^m}$. We 
view $x \in  {S}^{m- 1}$ and $y$ as a unit tangent vector in ${T}_{x}{S}^{m- 1}$. Thus ${V}_{2}( {\RR^m})$ 
becomes the unit vectors in the tangent bundle $T{S}^{m- 1}$. It is a smooth submanifold of ${\B{R}}^{2m}$ 
via the embedding
\[
  {V}_{2}( {\RR^m})  \subset  T{S}^{m- 1} \subset  T( {\RR^m})  = {\B{R}}^{2m}.
\]

It is better for our own purpose however to consider the embedding
\begin{align}\label{eq:20-8}
  \varphi: {V}_{2}( {\RR^m})  \ra  {S}^{{2m}- 1} \subset  {\B{C}}^{m},\quad
  \varphi ( {x,y})  = \frac{1}{\sqrt{2}}( {x- {iy}}).
\end{align}

The manifold ${V}_{2}( {\RR^m})$ is called the \Index{Stiefel manifold} of (orthogonal) 2-frames 
in ${\RR^m}$ ; it is evidently compact. The group ${SO}(2)$ of rotation matrices
\[
  {R}_{\theta } = \begin{pmatrix} 
    \cos \theta & -\sin \theta \\  
    \sin \theta & \cos \theta  
  \end{pmatrix}
\]
acts (smoothly) on ${V}_{2}( {\RR^m})$ by
\[
  (x, y).R_\theta = ( 
      ( {\cos \theta }) x + ( {\sin \theta }) y,
    - ( {\sin \theta }) x + ( {\cos \theta }) y
  ).
\]

The orbit space $V_2(\RR^m)/{SO}(2)$ is identified with the space $\widetilde{G}_2(\RR^m)$ 
of oriented 2-dimensional linear subspaces of ${\RR^m}$ by associating to $( {x,y})  \in  {V}_{2}( {\RR^m})$ 
the subspace they span, oriented so as to make $(x, y)$ a positively oriented orthonormal basis. We leave it to the 
reader as an exercise to specify a smooth manifold structure on $\widetilde{G}_{2}( {\RR^m})$. The resulting 
manifold is the \Index{Grassmann manifold} of oriented 2-dimensional subspaces of $\RR^m$.

It is clear from \eqref{eq:20-8} that
\[
  \varphi ( {( {x,y}) {R}_{\theta }})  
  = ( {\cos \theta  + i\sin \theta }) \varphi ( {x,y})
\]
so when we identify ${S}^{1} \subset  \B{C}$ with ${SO}( 2)$, the action of ${SO}( 2)$ on ${V}_{2}( {\RR^m})$ 
and the action of ${S}^{1}$ on ${S}^{{2m}- 1}$ correspond under $\varphi$. This gives a commutative diagram \eqref{eq:20-9}
\begin{equation}\label{eq:20-9}
  \begin{tikzcd}
    V_2(\RR^m)\dar{\pi_0}\rar{\varphi} & S^{2m-1}\dar{\pi_1}\rar{} & \CC^m-\{0\}\dar{\pi}\\
    \widetilde{G}_2(\RR^m)\rar{\tilde{\varphi}} & \CP^{m-1}\rar{\text{id}} & \CP^{m-1}
  \end{tikzcd}
\end{equation}
where ${\pi }_{0}( {x,y})  = {\R{span}}_{\RR}( {x,y})$ and ${\pi }_{1}$ is the restriction of 
the canonical map
\[
  \pi: {\B{C}}^{m}- \{ 0\}  \ra  {\CP}^{m- 1}
\]

If we use homogeneous coordinates on ${\CP}^{m- 1}$ then
\begin{align}\label{eq:20-10}
  \widetilde{\varphi }( {{\pi }_{0}( {x,y}) })  = [  {x- {iy}}].
\end{align}

It is not difficult to see that $\widetilde{\varphi }$ is injective and that its image is a smooth 
submanifold of ${\CP}^{m- 1}$ of (real) dimension ${2m}- 4$ ; cf. Exercise \ref{exercise:20-3}. We note that ${\pi }_{0}$ 
is a fiber bundle; cf. Example \ref{example:15-2}. Complex conjugation on the homogeneous coordinates for ${\CP}^{m-1}$ 
is an involution, whose fixed set is the real projective space ${\CP}^{m- 1}$, all of whose homogeneous 
coordinates are real. Since in \eqref{eq:20-10}, $x$ and $y$ are linearly independent, $\widetilde{\varphi }( {{\pi }_{0}( {x,y}) })$ 
is never fixed under conjugation, so
\begin{align}\label{eq:20-11}
  \widetilde{\varphi } : \widetilde{G}_{2}( {\RR^m})  \ra  {\CP}^{m- 1}- \B{RP}^{m- 1} = {W}_{m}.
\end{align}

We show below that this map is a homotopy equivalence, so that
\[
  H^*( {\widetilde{G}_{2}( {\RR^m}) })  \cong  H^*( {{\CP}^{m- 1}- \B{RP}^{m- 1}}),
\]
and use this to calculate $H^*( {\widetilde{G}_{2}( {\RR^m}) })$.

We begin with a discussion of the group $\R{GL}_{2}^{ + }( \B{R})$ of real $2 \times  2$ matrices with positive 
determinant. The action of the multiplicative group ${\B{C}}^*$ on $\B{C} = {\B{R}}^{2}$ by complex multiplication 
identifies ${\B{C}}^*$ with a subgroup of $\R{GL}_{2}^{ + }( \B{R})$, where $a + {ib} \in  {\B{C}}^*$ corresponds to
\begin{align}\label{eq:20-12}
  \begin{pmatrix} 
    a & - b \\  
    b & a 
  \end{pmatrix} 
  \in  \R{GL}_{2}^{ + }( \B{R})
\end{align}

Let $Q \subset  \R{GL}_{2}^{ + }( \B{R})$ be the subset of positive definite symmetric matrices with 
determinant 1. The subgroup ${SO}( 2)  \subset  {\R{GL}_2^+}( \B{R})$ acts on $Q$ by conjugation.


\begin{lemma}\label{lemma:20-4}
  The map
  \[
    \psi: Q \times  {\B{C}}^* \ra  {\R{GL}_2^+}( \B{R}) ;\;( {A,a + {ib}}) 
    \mapsto  A\cdot\begin{pmatrix} a & - b \\  b & a \end{pmatrix}
  \]
  is a homeomorphism
\end{lemma}

\begin{proof}
  Let $B \in  {\R{GL}_2^+}( \B{R})$ with transpose ${B}^*$. Then $B{B}^*$ is positive definite, 
  and by the spectral theorem it has a unique square root ${( B{B}^*) }^{1/2}$ which commutes with $B{B}^*$. 
  This gives the polar decomposition
  \[
    B = {( B{B}^*) }^{1/2} \cdot  R,\;R = {( B{B}^*) }^{-1/2} \cdot  B,
  \]
  and $R{R}^* = I$, so $R = {R}_{\theta } \in  {SO}( 2)$. For $d = \det B = \det {( B{B}^*) }^{1/2}$
  and $A =$  ${d}^{-1}{( B{B}^*) }^{1/2} \in  Q$, we obtain $B = \psi ( {A,d{e}^{i\theta }})$. The polar 
  decomposition is unique. Indeed if ${B}_{1}{R}_{1} = {B}_{2}{R}_{2}$ with ${B}_{i}$ symmetric 
  and ${R}_{i} \in  {SO}( 2)$ then ${B}_{1}^{2} = {B}_{2}^{2}$, and square roots are unique. Hence $\psi$ 
  is a bijection. Its inverse is continuous, since $A,d$ and $R$ depend continuously on $B$.
\end{proof}


For symmetric, positive definite matrices one can form powers ${A}^{t}$ for any $t \in  \B{R}$, and we have

\begin{lemma}\label{lemma:20-5}
  The space $Q$ is contractible by the homotopy
  \[
    F:Q \times [0, 1] \ra Q;\;F( {A,t})  = {A}^{t}.
  \]
\end{lemma}

\begin{proof}
  This is again a consequence of the spectral decomposition. A matrix $A \in  Q$ has positive 
  eigenvalues $\lambda,{\lambda }^{-1}$ with say $\lambda  \geq  1$. Then $\lambda$ depends 
  continuously on $A$. The case $\lambda  = 1$ occurs only for the identity matrix $I$. If $A \neq  I$ 
  we can write
  \[
    A = {\lambda P} + {\lambda }^{-1}( {I- P})  = {\lambda }^{-1}I + ( {\lambda - {\lambda }^{-1}}) P,
  \]
  where $P$ is the orthogonal projection on the (1-dimensional) $\lambda$-eigenspace of $A$. Here $P$ depends 
  continuously on $A \in  Q- \{ I\}$. We define $F$ on $( {Q-\{ I\} })  \times  [  {0,1}]$ by
  \[
    F( {A,t})  = {\lambda }^{t}P + {\lambda }^{-t}( {I- P})  = {\lambda }^{-t}I + ( {{\lambda }^{t}- {\lambda }^{-t}}) P.
  \]
  
  Observing that each matrix entry in $( {{\lambda }^{t}- {\lambda }^{-t}}) P$ has numerical value 
  at most $\lambda - {\lambda}^{-1}$, we see that $F$ extends continuously to $Q \times [0,1]$ 
  by $F( {I,t})  = I$.
\end{proof}


\begin{proposition}\label{proposition:20-6}
  The map
  \[
    \widetilde{\varphi } : \widetilde{G}_{2}( {\RR^m})  \ra  {\CP}^{m- 1}- \B{RP}^{m- 1}
  \]
  from \eqref{eq:20-11} is a homotopy equivalence.
\end{proposition}

\begin{proof}
  We write ${W}_{m} = {\CP}^{m- 1}- \B{RP}^{m- 1}$, and consider the smooth map (cf. \eqref{eq:20-11})
  \begin{align*}
    \Phi  : {V}_{2}( {\RR^m})  \times  {\R{GL}_2^+}( \B{R})  \ra  {\pi }^{-1}( {W}_{m})\\
    \Phi \left(x, y, \begin{pmatrix} a & c \\  b & a \end{pmatrix}\right) = ( {{ax} + {by}}) - i( {{cx} + {dy}}).
  \end{align*}
  
  A point in ${\pi }^{-1}( {W}_{m})$ has the form $z = v- {iw}$, where $v$ and $w$ are linearly independent 
  vectors in ${\RR^m}$. The fiber ${\Phi }^{-1}( z)$ is in 1-1 correspondence with the orthonormal 
  bases $(x, y)$ of ${\R{span}}_{\RR}( {v,w})$ which determine the same orientation as $(v, w)$. There is a 
  global smooth section $S$ of $\Phi$, constructed using the Gram-Schmidt orthonormalization process. 
  The fibers of $\Phi$ are the orbits of the ${SO}( 2)$-action,
  \[
    (x, y, B)  \cdot  {R}_{\theta} = ( {( {x,y})  \cdot  {R}_{\theta },{R}_{\theta }^{-1}B}),
  \]
  so $\Phi$ induces a bijection from the set of orbits
  \[
    {V}_{2}( {\RR^m}) { \times  }_{{SO}( 2) }{\R{GL}_2^+}( \B{R})  \ra  {\pi }^{-1}( {W}_{m}).
  \]
  
  This bijection commutes with the ${\B{C}}^*$-actions, if we let ${\B{C}}^*$ act on the domain 
  by right multiplication in ${\R{GL}_2^+}( \B{R})$ and on ${\pi }^{-1}( {W}_{m})  \subset  {\B{C}}^{m}$ 
  by scalar multiplication. 
  
  This gives a commutative diagram where the horizontal maps are bijections
  \[\begin{tikzcd}
    V_2(\RR^m)\times_{SO(2)}\R{GL}_2^+(\RR)\dar{}\rar{} & \pi^{-1}(W_m)\dar{}\\
    (V_2(\RR^m))\times_{SO(2)}\R{GL}_2^+(\RR)/\CC^*\rar{} & W_m
  \end{tikzcd}\]

  Lemma \ref{lemma:20-4} gives a bijection $Q \cong  {\R{GL}_2^+}( \B{R}) /{\B{C}}^*$, so we may identify 
  the lower left-hand corner of the diagram with the quotient space
  \[
    X = {V}_{2}( {\RR^m}) { \times  }_{{SO}( 2) }Q
  \]  
  for the ${SO}( 2)$-action on ${V}_{2}( {\RR^m})  \times  Q$ defined by
  \[
    ( {x,y,A})\cdot{R}_{\theta } = ( {( {x,y})\cdot{R}_{\theta },{R}_{\theta }^{-1}A{R}_{\theta }}).
  \]
  
  Altogether we obtain a bijection $\Phi  : X \ra  {W}_{m}$ given by
  \begin{align}\label{eq:20-13}
    \widetilde{\Phi}\left(\left[(x, y), \begin{pmatrix}\alpha & \beta\\ \beta & \gamma\end{pmatrix}\right] \right)  
    = [ (\alpha x + \beta y) - i(\beta x + \gamma y)] .
  \end{align}
  
  Evidently $\widetilde{\Phi }$ is continuous in the quotient topology on $X$, in fact it is a homeomorphism. 
  In a neighborhood $V$ of any given point in ${W}_{m}$ the inverse ${\widetilde{\Phi }}^{-1}$ can be 
  written as the composition
  \[
    V\overset{s}
    { \ra  }{\pi }^{-1}( {W}_{m}) \overset{S}
    { \ra  }{V}_{2}( {\RR^m})  \times  {\R{GL}_2^+}( \B{R}) 
    \overset{\mathrm{{id}} \times  {\psi }^{-1}}
    { \ra  }{V}_{2}( {\RR^m})  \times  Q \times  {\B{C}}^*
    \overset{\rho\circ \text{pr}}
    { \ra  }X
  \]
  where $s$ is a local section given by Lemma \ref{lemma:14-4}, $S$ the global section of $\Phi$ mentioned above, 
  and ${\psi }^{-1}$ given by Lemma \ref{lemma:20-4}. Each map in the sequence is continuous, so ${\widetilde{\Phi }}^{-1}$ 
  is continuous. As a byproduct we find that the canonical map $\rho  : {V}_{2}( {\RR^m})  \times  Q \ra  X$ 
  has continuous local sections defined on open sets covering $X$.
  
  The conjugation action of ${SO}( 2)$ on $Q$ fixes the identity matrix $I \in  Q$, so one has the subspace
  \begin{align}\label{eq:20-14}
    \widetilde{G}_{2}( {\RR^m}) 
    \xla[\pi_0](\simee)
    {V}_{2}( {\RR^m}) { \times  }_{{SO}( 2) }\{ I\}  
    \subset  X
  \end{align}
  (cf. \eqref{eq:20-9}). Comparing \eqref{eq:20-10} and \eqref{eq:20-13}, we see that $\widetilde{\Phi } \circ  {\pi }_{0}^{-1}$ 
  is precisely equal to $\widetilde{\varphi }$. It remains to be shown that the inclusion map ${i}_{0}$ in \eqref{eq:20-14} is 
  a homotopy equivalence. There is an obvious retraction induced by the constant map $Q \ra  \{ I\}$
  \[
    r: X \ra  {V}_{2}( {\RR^m}) { \times  }_{{SO}( 2) }\{ I\}  \subseteq  X.
  \]
  
  Finally the required homotopy $H : X \times  [  {0,1}]   \ra  X$ between ${i}_{0} \circ  r$ and ${\R{id}}_{X}$ is induced by
  \[
    {\R{id}}_{{V}_{2}( {\RR}^{m}) } \times  F\colon {V}_{2}( {\RR}^{m})  \times  Q \times  [  {0,1}]   \ra  {V}_{2}( {\RR}^{m})  \times  Q,
  \]
  where $F$ is defined in Lemma \ref{lemma:20-5}. Observe that $H$ is well-defined 
  because $F( {{R}_{\theta }^{-1}A{R}_{\theta },t}) = {R}_{\theta }^{-1}F( {A,t}) {R}_{\theta }$. 
  Continuity of $H$ can be shown with use of local sections of $\rho$.
\end{proof}


\begin{remark}\label{remark:20-7}
  We can offer the following more conceptual explanation of the construction in the previous proof. 
  Consider the set ${G}_{2}^{\B{C}}( {\RR^m})$ of pairs $(V, J)$ where $V \subseteq  {\RR^m}$ is a 2-dimensional 
  real oriented linear subspace and $J$ a complex structure on $V$ compatible with the orientation, i.e. 
  an $\B{R}$-linear map $J:V \ra V$ with ${J}^{2} = - \mathrm{{id}}$ and such that $(x, Jx)$ for $x \in  V,x \neq  0$ 
  is a positively oriented basis for $V$. One forms the complexifications ${V}_{\B{C}} = V\,{ \otimes  }_{\B{R}}\,\B{C} 
  \subset  {\B{C}}^{m}$ and ${J}_{\B{C}} = J\,{ \otimes  }_{\B{R}}\,{\R{id}}_{\B{C}} : {V}_{\B{C}} \ra  {V}_{\B{C}}$. 
  Then ${V}_{\B{C}} = {V}_{ + } \oplus  {V}_{- }$, where ${V}_{ \pm  }$ are the $( {\pm i})$- eigenspaces of ${J}_{\B{C}}$. 
  These are 1-dimensional over $\B{C}$. In fact for $x \in  V- \{ 0\}$,
  \[
    {V}_{ + } = {\R{span}}_{\B{C}}( {x- {iJx}}),\;{V}_{- } = {\R{span}}_{\B{C}}( {x + {iJx}})
  \]
  
  The pair $(V, J)$ may be recovered from ${V}_{ + }$ since ${V}_{- }$ is complex conjugate to $V_+$, $V= (V_+-V_{-})\cap \RR^m$ 
  and since $J$ is the restriction of the $\B{C}$-linear endomorphism of ${V}_{ + } \oplus  {V}_{- }$ which acts on ${V}_{ \pm  }$ 
  by multiplication with $\pm  i$. This gives an identification of ${G}_{2}^{\B{C}}( {\RR^m})$ with ${W}_{m}$ in which $(V, J)$ 
  corresponds to ${V}_{ + }$ considered as a point in ${W}_{m}\subseteq {\CP}^{m-1}$.
  
  The space $Q \cong  {\R{GL}_2^+}( \B{R}) /{\B{C}}^*$ parametrizes the complex structures on ${\B{R}}^{2}$ 
  compatible with the standard orientation (cf. Exercise \ref{exercise:20-5}) so $X = {V}_{2}( {\RR^m}) { \times  }_{{SO}( 2) }Q$ is 
  another model for ${G}_{2}^{\mathrm{C}}( {\RR^m})$. The homeomorphism $\widetilde{\Phi }$ in \eqref{eq:20-13} is the natural 
  identification of the two models.
  
  Each $V \in  \widetilde{G}_{2}( {\RR^m})$ has a canonical complex structure ${J}_{0}$ such that any unit 
  vector $x \in  V$ leads to a positively oriented orthonormal basis $( {x,{J}_{0}x})$ for $V$. The 
  inclusion $\widetilde{G}_{2}( {\RR^m})  \ra  {G}_{2}^{\B{C}}( {\RR^m})$, which sends $V$ to $( {V,{J}_{0}})$, 
  corresponds exactly to the embedding $\widetilde{\varphi }$ in \eqref{eq:20-10}.
\end{remark}


We shall now use Proposition \ref{proposition:20-6} to calculate the cohomology of $\widetilde{G}_{2}( {\RR^m})$. 
First we apply Poin’caré duality to evaluate $H^*( {W}_{m})$. Consider the inclusions
\[
  i : {W}_{m} \ra  {\CP}^{m- 1},\;j : \B{RP}^{m- 1} \ra \B{CP}^{m- 1}.
\]

When $m = 2,{S}^{2}\overset{ \cong  }{ \ra  }{\CP}^{1}$ by Example \ref{example:14-1}, and $\B{RP}^{1} \subset  {\CP}^{1}$ 
becomes identified with a great circle in ${S}^{2}$ so that ${W}_{2}$ becomes the disjoint union of two open 
hemispheres in ${S}^{2}$.

\begin{lemma}\label{lemma:20-8}
  The map ${i}_{ * }\!: H_{c}^{q}( {W}_{m})  \ra  H^{q}( {\CP}^{m- 1})$ is an isomorphism except in 
  possibly two cases: for $q = 0$, and, if $m$ is even, for $q = m$. In fact $H_{c}^{0}( {W}_{m})  = 0$, 
  and for $m$ even there is a short exact sequence
  \[
    0 \ra  \B{R} \ra  H_{c}^{m}( {W}_{m}) \overset{{i}_{ * }}{ \ra  }H^{m}( {\CP}^{m- 1})  \ra  0.
  \]
\end{lemma}

\begin{proof}
  The exact sequence of Proposition \ref{proposition:13-11} for the pair $( {{\CP}^{m- 1}, \B{RP}^{m- 1}})$ takes the form
  \[
    \cdots \overset{{j}^*}
    { \ra  }H^{q- 1}( \B{RP}^{m- 1}) 
    \overset{\delta }{ \ra  }
    H_{c}^{q}( {W}_{m}) 
    \overset{{i}_{ * }}{ \ra  }
    H^{q}( {\CP}^{m- 1}) 
    \overset{j * }{ \ra  }H^{q}( \B{RP}^{m- 1}) 
    \overset{\delta }{ \ra  }
    \cdots
  \]
  
  The terms involving $\B{RP}^{m-1}$ and ${\CP}^{m- 1}$ have been calculated in Example \ref{example:9-31} and Theorem \ref{theorem:14-2}, 
  respectively. Note that $H^{0}({\CP}^{m- 1})\cong\B{R}$ maps isomorphically to $H^{0}( \B{RP}^{m- 1})\cong\B{R}$ 
  under ${j}^*$, whereas ${j}^* = 0$ in other degrees.
\end{proof}

\begin{proposition}\label{proposition:20-9}
  The cohomology $H^{{2p}- 1}( {W}_{m})  = 0$ for all $p$, and
  \[
    H^{2p}( {W}_{m})  \cong  
    \left\{\!\!\begin{array}{ll} 
      {\B{R}}^{2} & \text{ if }{2p} = m- 2 \\  
      \B{R} & \text{ if }{2p} \neq  m- 2\text{ and }0 \leq  {2p} \leq  {2m}- 4 \\ 
      0 & \text{ if }{2p} \geq  {2m}- 2. 
    \end{array}\right.
  \]
  
  Moreover $H^{q}( i)  : H^{q}( {\CP}^{m- 1})  \ra  H^{q}( {W}_{m})$ is a monomorphism if $q \neq  {2m}- 2$, and 
  is zero if $q = {2m}- 2$.
\end{proposition}

\begin{proof}
  We apply Poincaré duality to the oriented $(2m- 2)$-dimensional manifold ${\CP}^{m- 1}$ and to ${W}_{m}$. 
  Lemma \ref{lemma:13-6} gives a commutative diagram
  \[\begin{tikzcd}
    H^p(\CP^{m-1})\dar{\simee}\rar{H^p(i)} & H^p(W_m)\dar{\simee}\\
    H_{c}^{2m-2-p}(\CP^{m-1})\rar{i^!} & H_{c}^{2m-2-p}(W_m)^*
  \end{tikzcd}\]
  
  The previous lemma implies that $H^{p}( i)$ is an isomorphism except if $p = {2m}- 2$, or 
  if $p = m- 2$ and $m$ is even, because ${i}^{!}$ is the vector space dual of ${i}_{ * }$. 
  In the first case
  \[
    H^{{2m}- 2}( {W}_{m})  \cong  H_{c}^{0}{( {W}_{m}) }^* = 0
  \]
  but $H^{{2m}- 2}( {\CP}^{m- 1})  \cong  \B{R}$. In the second case, the exact sequence of Lemma \ref{lemma:20-8} 
  dualizes to the exact sequence
  \[
    0 \ra  H^{m- 2}( {\CP}^{m- 1}) \overset{H^{m- 2}( i) }{ \ra  }H^{m- 2}( {W}_{m})  \ra  \B{R} \ra  0,
  \]
  and the proposition follows from Theorem \ref{theorem:14-2}.
\end{proof}

We have the two canonical bundles ${\gamma }_{2}$ and ${\gamma }_{2}^{ \bot  }$ over $\widetilde{G}_{2}( {\RR^m})$ 
with total spaces
\begin{align*}
  E( {\gamma }_{2}) & = \left\{  {( {V,v})  \in  \widetilde{G}_{2}( {\RR^m})  \times  {\RR^m} \mid  v \in  V}\right\}\\
  E( {\gamma }_{2}^{ \bot  }) & = \left\{  {( {V,v})  \in  \widetilde{G}_{2}( {\RR^m})  \times  {\RR^m} \mid  v \in  {V}^{ \bot  }}\right\}  
\end{align*}
and with ${\gamma }_{2} \oplus  {\gamma }_{2}^{ \bot  } = {\varepsilon }^{m}$. Alternatively $E( {\gamma }_{2})  = {V}_{2}( {\RR^m}) { \times  }_{{SO}( 2) }{\B{R}}^{2}$, 
the orbit space of the ${SO}( 2)$-action given by $( {x,y,v}) {R}_{\theta } = ( {( {x,y}) {R}_{\theta },{R}_{\theta }^{-1}v})$. 
The embedding of ${V}_{2}( {\RR^m}) { \times  }_{{SO}( 2) }{\B{R}}^{2}$ into $\widetilde{G}_{2}( {\RR^m})  \times  {\RR^m}$ 
sends $[  {x,y,v}]$ into $( {{\pi }_{0}( {x,y}),x{v}_{1} + y{v}_{2}})$ where $v = ( {{v}_{1},{v}_{2}})$.

There is a bundle map over $\widetilde{\varphi }$ from ${\gamma }_{2}$ to the underlying real bundle $H_{\mathrm{R}}$ of 
the canonical line bundle over ${\CP}^{m- 1}$,
\[\begin{tikzcd}
  V_2(\RR^m)\times_{SO(2)}\RR^2\rar{\varphi\times\id}\dar{} & S^{2m-1}\times_{S^1}\CC\dar{}\\
  \widetilde{G}_2(\RR^m)\rar{\tilde{\varphi}} & \CP^{m-1}
\end{tikzcd}\]

Since ${c}_{1}( H)  = e( H_{\RR})$ by Theorem \ref{theorem:19-6}.(ii), naturality of the Euler class 
gives $e( {\gamma }_{2})  = {\widetilde{\varphi }}^*( {{c}_{1}( H) })$

Let $c \in  H^{2}( {\widetilde{G}_{2}( {\RR^m}) })$ be the Euler class of ${\gamma }_{2},c = e( {\gamma }_{2})$. 
If $m$ is even, we let $e = e(\gamma_{2}^\bot)$ be the Euler class in $H^{m-2}(\widetilde{G}_{2}(\RR^m))$.

\begin{theorem}\label{theorem:20-10}
  With the notation above
  \begin{enumerate}
    \item For $m$ odd and $m \geq  3,H^*( {\widetilde{G}_{2}( {\RR^m}) })  = \B{R}[  c]  /( {c}^{m- 1})$
    \item For $m$ even and $m \geq  4$,
      \[
        H^*( {\widetilde{G}_{2}( {\RR^m}) })  = \B{R}[  {c,e}]  /( {{c}^{m- 1},{ce},{e}^{2} + {(-1) }^{m/2}{c}^{m- 2}})
      \]
      with $\deg c = 2$ and $\deg e= m-2$.
  \end{enumerate}
\end{theorem}

\begin{proof}
  We already know the additive structure by Propositions \ref{proposition:20-6} and \ref{proposition:20-9}, 
  and also that
  \[
    H^*( \widetilde{\varphi })  : H^*( {\CP}^{m- 1})  \ra  H^*( {\widetilde{G}_{2}( {\RR^m}) })
  \]
  has kernel $H^{{2m}- 2}( {\CP}^{m- 1})$ and is onto for odd $m$. Since
  \[
    H^*( {\CP}^{m- 1})  = \RR[  {{c}_{1}( H) }]  /( {{c}_{1}{( H) }^{m}})
  \]
  and $c = H^*( \widetilde{\varphi }) ( {{c}_{1}( H) })$, this proves (i).
  
  Suppose now $m = {2n} \geq  4$. We first establish the relations:
  \begin{align}\label{eq:20-15}
    {c}^{m- 1} = 0,\;{ce} = 0,\;{e}^{2} + {(-1) }^{n}{c}^{m- 2} = 0.
  \end{align}
  
  Indeed, the first one follows from Proposition \ref{proposition:20-9}, since $c$ and hence any power of $c$ is in the image of
  \[
    H^*( {\CP}^{m- 1}) 
    \overset{H^*( i) }{ \ra  }
    H^*( {W}_{m}) 
    \overset{H^*( \widetilde{\varphi }) }{ \ra  }
    H^*( {\widetilde{G}_{2}( {\RR}^{m}) }).
  \]
  
  The second relation is a consequence of Theorem \ref{theorem:19-6}.(ii),
  \[
    {ec} 
    = e( {\gamma }_{2}) e( {\gamma }_{2}^{ \bot  })  
    = e( {{\gamma }_{2} \oplus  {\gamma }_{2}^{ \bot  }})  
    = e( {\varepsilon }^{m})  
    = 0.
  \]
  
  For the third relation we use that the total Pontryagin class is exponential (cf. (18.\ref{eq:18-14})) so that
  \[
    ( {1 + {p}_{1}( {\gamma }_{2}) }) 
    ( 
      {1 + {p}_{1}( {\gamma }_{2}^{ \bot  })  
      + \ldots  + {p}_{n- 1}( {\gamma }_{2}^{ \bot  }) }
    ) = 1
  \]
  and hence ${p}_{j}( {\gamma }_{2}^{ \bot  })  = {(-1) }^{j}{p}_{1}{( {\gamma }_{2}) }^{j}$ for $j \leq  n- 1$. 
  Moreover by Proposition \ref{proposition:19-9}
  \[
    {e}^{2} 
    = e{( {\gamma }_{2}^{ \bot  }) }^{2} 
    = {p}_{n- 1}( {\gamma }_{2}^{ \bot  })  
    = {(-1) }^{n- 1}{p}_{1}{( {\gamma }_{2}) }^{n- 1}.
  \]
  
  Since ${\gamma }_{2} = {\widetilde{\phi }}^*( H_{\RR})$ and $H_{\RR\CC} = H \oplus  H^*$,
  \[
    {p}_{1}( {\gamma }_{2})  
    = - {\widetilde{\phi }}^*( {{c}_{2}( {H \oplus  H^*}) })  
    = {\widetilde{\phi }}^*( {{c}_{1}{( H) }^{2}})  
    = {c}^{2}
  \]
  so ${e}^{2} = {(-1) }^{n- 1}{c}^{m- 2}$, which is the last equation of \eqref{eq:20-15}. From Proposition \ref{proposition:20-9} 
  and the non-triviality of ${c}_{1}{( H) }^{m- 2}$ we know that ${c}^{m- 2} \neq  0$, so by \eqref{eq:20-15} also 
  that $e \neq  0$. The vector space $H^{m- 2}( {\widetilde{G}_{2}( {\RR^m}) })$ is 2-dimensional, 
  and ${c}^{n- 1} = {\lambda e},\lambda  \in  \RR- \{ 0\}$ gives ${c}^{n} = {\lambda ec} = 0$, which contradicts 
  that $m-$  $2 \geq  n$ and ${c}^{m- 2} \neq  0$. We have proved that the set $\left\{  {1,c,\ldots,{c}^{m- 2},e}\right\}$ 
  is a vector space basis for $H^*( {\widetilde{G}_{2}( {\RR^m}) })$, but this is also a basis for the 
  ring $\B{R}[  {c,e}]  /( {{c}^{m- 1},{ce},{e}^{2} + {(-1) }^{n}{c}^{m- 2}}).$
\end{proof}


Let $\zeta$ be a smooth $m$-dimensional oriented real vector bundle over $M$, and suppose $\zeta$ has an 
inner product. Consider the associated smooth fiber bundle $\widetilde{G}_{2}( \zeta )$ over $M$ with total space
\[
  E( {\widetilde{G}_{2}( \zeta ) })  = \left\{  {( {p,V})  \mid  p \in  M,V \in  \widetilde{G}_{2}( {\zeta }_{p}) }\right\}
\]
and with $\pi  : \widetilde{G}_{2}( \zeta )  \ra  M$ being the projection onto the first factor. There are 
two oriented vector bundles over $\widetilde{G}_{2}( \zeta )$ with total spaces
\begin{align*}
  E( {{\gamma }_{2}( \zeta ) })  
    & = \left\{  {( {p,V,v})  \mid  ( {p,V})  \in  \widetilde{G}_{2}( \zeta ),v \in  V}\right\}\\
  E( {{\gamma }_{2}^{ \bot  }( \zeta ) })  
    & = \left\{  {( {p,V,v})  \mid  ( {p,V})  \in  \widetilde{G}_{2}( \zeta ),v \in  {V}^{ \bot  }}\right\}
\end{align*}
and ${\gamma }_{2}( \zeta )  \oplus  {\gamma }_{2}^{ \bot  }( \zeta )  \cong  {\pi }^*( \zeta )$. The orientation 
of ${\gamma }_{2}^{ \bot  }( {\zeta }_{p})$ is such that ${\gamma }_{2}( {\zeta }_{p})  \oplus  {\gamma }_{2}^{ \bot  }( {\zeta }_{p})$ 
has the same orientation as ${\zeta }_{p}$. An application of Theorem \ref{theorem:20-1} gives


\begin{theorem}\label{theorem:20-11}
  For any oriented $m$-dimensional real vector bundle $\zeta,H^*( {\widetilde{G}_{2}( \zeta ) })$ is 
  a free $H^*( M)$ module with basis
  \[
    \left\{  \begin{array}{ll} 1,e( {{\gamma }_{2}( \zeta ) }),\ldots,e{( {\gamma }_{2}( \zeta ) ) }^{m- 2},e( {{\gamma }_{2}^{ \bot  }( \zeta ) }) & \text{ if }m = {2n} \geq  4 \\  1,e( {{\gamma }_{2}( \zeta ) }),\ldots,e{( {\gamma }_{2}( \zeta ) ) }^{m- 2} & \text{ if }m = {2n} + 1 \geq  3 \end{array}\right.
  \]
  In particular ${\pi }^* : H^*( M)  \ra  H^*( {\widetilde{G}_{2}( \zeta ) })$ is injective.
\end{theorem}

Given this result, the proof of the real splitting principle, Theorem \ref{theorem:19-7}, is precisely analogous to the 
proof of the complex splitting principle, treated earlier in this chapter.