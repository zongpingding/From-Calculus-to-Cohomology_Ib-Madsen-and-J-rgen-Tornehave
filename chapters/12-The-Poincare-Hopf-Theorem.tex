\chapter{The Poincar\'e-Hopf Theorem }
In the following, $M^n\subseteq\RR^{n+k}$ will denote a fixed Smooth-submanifold. If the
cohomology of $M^n$ is finite-dimensional (e.g. when Mn is compact), the $i$-th
\Index{Betti number} is given by
\begin{align}\label{eq:12-1}
  b_i(M) = \dim H^i(M^n).
\end{align}

The \Index{Euler characteristic} of $M^n$ is defined to be
\begin{align}\label{eq:12-2}
  \chi(M) = \sum_{i=0}^{n }{(-1)^i b_i(M)}.
\end{align}

This chapter's main result is:
\begin{theorem}[Poincar\'{e}-Hopf]\label{theorem:12-1}\index{Poincar\'{e}-Hopf theorem}
  Let $X$ be a smooth vector field on a compact manifold $M$. If $X$ has only isolated zeros then
  \[
    \R{Index}(X) = \chi(M).
  \]
\end{theorem}

By the final result of Chapter 11 it is sufficient to show the formula for just one
such vector field $X$ on $M$. We shall do so by making use of a Morse function on $M^n$.

Given $f\in C^\infty(M, \RR)$, a point $p\in M$ is a critical point for $f$ if $\dd_pf=0$.

\begin{proposition}\label{proposition:12-2}
  Suppose that $p\in M$ is a critical point for $f\in C^\infty(M, \RR)$.
  \begin{enumerate}[(i)]
    \item There exsits a quadratic form $\dd_p^2f$ on $T_pM$ characterized by the equation
      \[
        \dd_p^2f(\alpha'(0)) = (f\circ\alpha)''(0)
      \]
      where $\alpha:(-\delta,\delta)\to M$ is any smooth curve with $\alpha(0)=p$.
    \item Let $h:U\to\RR^n$ be a chart around $p$ and let $q=h(p)$. Then the composition 
      \[
        \RR^n\xra[D_qh^{-1}] T_pM\xra[\dd_p^2f] \RR
      \]
      is the quadratic fonn associated to the symmetric matrix
      \[
        \left(\frac{\partial^2f\circ h^{-1}}{\partial x_i\partial x_j}(q)\right)
      \]
  \end{enumerate}
\end{proposition}

\begin{proof}
  Set $h\circ\alpha(t) = \gamma(t) = (\gamma_1(t), \ldots, \gamma_n(t))$ and $\phi = f\circ h^{-1}$. A direct 
  calculation yields
  \[
    (f\circ\alpha)'(t)
    = (\phi\circ\gamma)'(t)
    = \sum_{i=1}^{n}{\frac{\partial\phi}{\partial x_i}(\gamma(t))\gamma_i'(t)}.
  \]
  Since $p$ is critical, $(\partial\phi/\partial x_i)(\gamma(0)) = 0$. By differentiating once again and substituting 
  $t=0$, we get 
  \[
    (f\circ\alpha)''(0)
    = \sum_{i=1}^n\sum_{j=1}^n{\frac{\partial^2\phi}{\partial x_i\partial x_j}(q)\gamma_i'(0)\gamma_j'(0)}.
  \]
  This is the value at $\gamma'(0) = D_ph(\alpha'(0))\in\RR^n$ of the quadratic form from (ii).
  Both (i) and (ii) follow.
\end{proof}

Consider another chart $\tilde{h}:\tilde{U}\to\RR^n$ around $p$ with $\tilde{q} = \tilde{h}(p)$ and let $F=\tilde{h}\circ h^{-1}$
defined in a neighborhood of $q$. The last formula in the proof above can be compared with 
\[ 
  (f\circ\alpha)''(0)
  = \sum_{i=1}^n\sum_{j=1}^n{\frac{\partial^2\tilde{\phi}}{\partial x_i\partial x_j}(\tilde{q})\tilde\gamma_i'(0)\tilde\gamma_j'(0)}.
\]

where $\tilde{\phi} = f\circ\tilde{h}^{-1}$ and $\tilde{\gamma}=\tilde{h}\circ\alpha$. Let $J$ denote the Jacobian matrix associated
with $F$ in $q$. Then $\tilde{\gamma}'(0)=J\gamma'(0)$ for the column vectors $\tilde{\gamma}'(0)$ and $\gamma'(0)$. 
By substituting this and comparing, one obtains the matrix identity
\begin{align}\label{eq:12-3}
  \left(\frac{\partial^2\phi}{\partial x_i\partial x_j}(q)\right) 
  = J^t\left(\frac{\partial^2\tilde{\phi}}{\partial x_i\partial x_j }(\tilde{q})\right)J.
\end{align}

\begin{definition}\label{definition:12-3}
  A critical point $p\in M$ of $f\in C^\infty(M, \RR)$ is said to be non-degenerate, if the matrix in 
Proposition \ref{proposition:12-2}.(ii) is invertible. We call $f$ a \Index{Morse function}, if all critical points of $f$ are 
non-degenerate. The \Index{index} of a \Index{non-degenerate critical point} $p$ is the maximal dimension of a subspace $V\subseteq T_pM$ for
which the restriction of $d_p^2f$ to $V$ is negative definite.
\end{definition}

For smooth submanifolds $M^n\subseteq\RR^{n+k}$ one can get Morse functions by:
\begin{theorem}\label{theorem:12-4}
  For almost all $p_0\in\RR^{n+k}$ the function $f:M\to\RR$ defined by
  \[
    f(p) = \frac{1}{2}\|p-p_0\|^2
  \]
  is a Morse function.
\end{theorem}

\begin{proof}
  Let $g:\RR^n\to M$ be a local parametrization and $Y_j:\RR^n\to\RR^{n+k},j=1,\ldots,k$ smooth maps, such 
  that $Y_1(\F x), \ldots, Y_k(\F x)$ is a basis of $T_{g(\F x)}M^\perp$ for all $\F x\in\RR^n$ By Lemma \ref{lemma:9-21} we know that $M$ 
  can be covered by at most countably many coordinate patches $g(U)$ of this type. Therefore it suffices to prove the
  assertion for $g(U)$ instead of $M$.

  We define $\Phi:\RR^{n+k}\to\RR^{n+k}$ by 
  \begin{align}\label{eq:12-4}
    \Phi(\F x, \F t) = g(\F x) + \sum_{j=1}^k{t_jY_j(\F x)}\qquad (\F x\in\RR^n, \F t\in\RR^k). 
  \end{align}
  By Sard's theorem it suffices to prove that if $p_0$ is regular value of $\Phi$, then $f$ becomes 
  a Morse function on $g(U)$. Set $k=f\circ g:\RR^n\to\RR^n$; we can show instead that $k$ becomes 
  a Morse function on $\RR^n$. We have 
  \begin{align}\label{eq:12-5}
    k(\F x) = \frac12\langle g(\F x)-p_0, g(\F x)-p_0\rangle,
  \end{align}
  where $\langle\cdot,\cdot\rangle$ denotes the standard inner product in $\RR^{n+k}$. 

  Since $\langle \partial g/\partial x_i(\F x), Y_\nu(\F x)\rangle = 0$, it follows by Differentiation with respect to 
  $x_j$ that 
  \begin{align}\label{eq:12-6}
    \biggl\langle \frac{\partial^2 g}{\partial x_i\partial x_j}, Y_\nu\biggr\rangle 
    = -\biggl\langle \frac{\partial g}{\partial x_i}, \frac{\partial Y_\nu}{\partial x_j}\biggr\rangle.
  \end{align}
  From \eqref{eq:12-5} we have 
  \begin{align}\label{eq:12-7}
    \frac{\partial k}{\partial x_i } = \biggl\langle g(\F x)-p_0, \frac{\partial g}{\partial x_i}\biggr\rangle,
  \end{align}
  and therefore 
  \begin{align}\label{eq:12-8}
    \frac{\partial^2 k}{\partial x_i\partial x_j} 
    = \biggl\langle \frac{\partial g}{\partial x_j}, \frac{\partial g}{\partial x_i}\biggr\rangle
    + \biggl\langle g(\F x)-p_0, \frac{\partial^2 g}{\partial x_i\partial x_j}\biggr\rangle.
  \end{align}
  Furthermore, by \eqref{eq:12-4},
  \begin{align}\label{eq:12-9}
    \frac{\partial \Phi}{\partial x_j}
    = \frac{\partial g}{\partial x_j}
      + \sum_{\nu=1}^k{t_\nu\frac{\partial Y_\nu}{\partial x_j}},\qquad
    \frac{\partial \Phi}{\partial t_\nu} = Y_\nu.
  \end{align}
  Assume that $p_0$ is a regular value of $\Phi$ and let $\F x$ be a critical point of $k$. It follows
  from \eqref{eq:12-7} that $g(x)-p_0\in T_{g(\F x)}M^\perp$. Hence there exists a unique $t\in\RR^k$ with
  \begin{align}\label{eq:12-10}
    g(\F x) - p_0 = -\sum_{\nu=1}^k{t_\nu Y_\nu(\F x)},
  \end{align}
  and $(\F x, \F t)\in\Phi^{-1}(p_0)$. The $n+k$ vectors in \eqref{eq:12-9} are linearly independent at the 
  point $(\F x, \F t)$. At this point, the equations \eqref{eq:12-8}, \eqref{eq:12-10} \eqref{eq:12-6} and \eqref{eq:12-9} 
  give 
  \begin{align*}
    \frac{\partial^2k}{\partial x_i\partial x_j}
    & = \biggl\langle \frac{\partial g }{\partial x_j}, \frac{\partial g}{\partial x_i}\biggr\rangle
      - \biggl\langle \sum_{\nu=1}^{k}{t_\nu Y_\nu}, \frac{\partial^2 g}{\partial x_i\partial x_j}\biggr\rangle\\
    & = \biggl\langle \frac{\partial g}{\partial x_j}, \frac{\partial g}{\partial x_i}\biggr\rangle
      + \biggl\langle \sum_{\nu=1}^{k}{\frac{\partial Y_\nu}{\partial x_j}}, \frac{\partial g}{\partial x_i} \biggr\rangle 
      = \biggl\langle \frac{\partial \Phi}{\partial x_j}, \frac{\partial g}{\partial x_i}\biggr\rangle.
  \end{align*}
  Let $A$ denote the invertible $(n+k)\times(n+k)$ matrix with bectors from \eqref{eq:12-9} as rows. 
  Then $AD_{\F x}g$ takes the following form:
  \[
    \begin{pmatrix}
      \frac{\partial^2 k}{\partial x_1\partial x_1} & \cdots & \frac{\partial^2 k}{\partial x_1\partial x_n}\\
      \vdots &  & \vdots\\
      \frac{\partial^2 k}{\partial x_n\partial x_1} & \cdots & \frac{\partial^2 k}{\partial x_n\partial x_n}\\
      0 & \cdots & 0 \\
      \vdots &  & \vdots\\
      0 & \cdots & 0
    \end{pmatrix}
  \]
  Since $D_xg$ has rank $n$, so does $AD_{\F x}g$. Hence the $n\times n$ matrix 
  \[
    \left(\frac{\partial^2 k}{\partial x_i\partial x_j}(\F x)\right)
  \]
  is invertible. This shows that $\F x$ is a non-degenerate critical point.
\end{proof}

\begin{example}\label{example:12-5}
  Let $f:\RR^n\to\RR$ be the function
  \[
    f(x) = c - x_1^2 - x_2^2 - \cdots - x_\lambda^2 + x_{\lambda+1}^2 + \cdots + x_n^2,
  \]
  where $x\in\RR, \lambda\in\ZZ$ and $0\le\lambda\le n$. Since 
  \[
    \grad_{\F x}(f) = 2(-x_1, -x_2, \ldots, -x_\lambda, x_{\lambda+1}, \ldots, x_n),
  \]
  0 is the only critical point of $f$. We find that 
  \[
    \left(\frac{\partial^2 f}{\partial x_i\partial x_j}(0)\right)
    = \R{diag}(-2, -2, \ldots, -2, 2, \ldots, 2)
  \]
  with exactly $\lambda$ diagonal entries equal to -2. Thus the origin is non-degenerate
  of index $\lambda$. We note that the vector field $\grad(f)$ has the origin as its only zero
  and that it is non-degenerate of index $(-1)^\lambda$.
\end{example}

\begin{theorem}\label{theorem:12-6}
  Let $p\in M^n$ be a non-degenerate critical point for $f\in C^\infty(M,\RR)$. There exists a $C^\infty$-chart 
  $h:U\to h(U)\subseteq\RR^n$ with $p\in U$ and $h(p)=0$ such that 
  \[
    f\circ h^{-1}(\F x) = f(p) + \sum_{i=1}^n{\delta_i x_i^2},\qquad \F x\in h(U),
  \]
  where $\delta_i=\pm1 (1\le i\le n)$ (By an additional permutation of coordinates we can put $f$ into the 
  standard form given in Example \ref{example:12-5}.)
\end{theorem}

\begin{proof}
  After replacing $f$ with $f - f(p)$ we may assume that $f(p) = 0$. Since
the problem is local and diffeomorphism invariant, we may also assume that
$f\in C^\infty(W, \RR)$, where $W$ is an open convex neighborhood of 0 in $\RR^n$ and that 0
is the considered non-degenerate critical point with $f(0) = 0$.

We write $f$ in the form
\[
  f(\F x) = \sum_{i=1}^{n }{x_i g_i(\F x)};\qquad g_i(\F x) = \int_0^1 \frac{\partial f(t\F x)}{\partial x_i}\dd t.
\]
Since $g_i(0) = \frac{\partial f }{\partial x_i}(0) = 0$, we may repeat to get 
\[
  g_i(\F x) = \sum_{j=1}^{n }{x_j g_{ij}(\F x)};\qquad 
  g_{ij}(\F x) = \int_0^1 \frac{\partial g_i(s\F x)}{\partial x_j}\dd s.
\]
On $W$ we now have that 
\[
  f(\F x) = \sum_{i=1}^{n }{\sum_{j=1}^{n }{x_ix_j g_{ij}(\F x)}}.
\]
where $g_{ij}\in C^\infty(W, \RR)$. If we introduce $h_{ij}=\frac12(g_{ij}+g_{ji})$ then $(h_{ij})$ becomes 
a symmetric $n\times n$ matrix of smooth functions on $W$, and 
\begin{align}\label{eq:12-11}
  f(\F x) = \sum_{i=1}^{n }{\sum_{j=1}^{n }{x_ix_j h_{ij}(\F x)}}.
\end{align}
By differentiating (II) twice and substituting 0, we get
\[
  \frac{\partial^2 f}{\partial x_i\partial x_j}(0) = 2h_{ij}(0).
\]
In particular the matrix $(h_{ij}(0))$ is invertible.

Let us return to the original $f\in C^\infty(M, \RR)$. By induction on $k$, we attempt to
show that the $C^\infty$-chart $h$ from the theorem can be choosen such that $f\circ h^{-1}$
is given by \eqref{eq:12-11} with
\[
  (h_{ij}) = \begin{pmatrix}
    D & 0\\
    0 & E
  \end{pmatrix}
\]
with $D$ a $(k - 1)\times(k - 1)$ matrix of the form $\R{diag}(\pm1,\ldots, \pm1)$, and $E$ some
symmetric $(n-k+1)\times(n-k+1)$ matrix of smooth functions. So suppose inductively that
\begin{align}\label{eq:12-12}
  f(\F x) = \sum_{i=1}^{k-1}{\delta_ix_i^2} + \sum_{i=k}^{n}{\sum_{j=k}^{n}{x_ix_jh_{ij}(\F x)}}
  ,\qquad \delta_i = \pm1
\end{align}
for $\F x$ in a neighborhood $W$ of the origin. We know that the minor $E$ is invertible
at 0. To start off we can perfonn a linear change of variables in $X_k,\ldots,X_n$, so
that our new variables satisfy \eqref{eq:12-12} with $h_{kk}(0)\neq 0$. By continuity we may assume
that $h_{kk}(\F x)$ has constant sign $\delta_k = \pm1$ on the entire $W$. Set
\[
  q = \sqrt{|h_{kk}|}\in C^\infty(W, \RR),
\]
and introduce new variables:
\begin{align*}
  & y_k = q(\F x)\biggl(x_k + \sum_{i=k+1}^{n }{x_i\frac{h_{ik}(\F x)}{h_{kk}(\F x)}}\biggr) \\
  & y_j = x_j \qquad \text{ for } j\neq k, 1\le j\le n. 
\end{align*}
The Jacobi detenninant for $\F y$ as function of $\F x$ is easily seen to be $\partial y_k/\partial x_k(0)=q(0)\neq 0$. 
The change of variables thus defines a local diffeomorphism $W$ around 0. In a neighborhood around 0 we have for$ y = \Psi(\F x)$:
\begin{align*}
  f\circ\Psi^{-1}(\F y)
  & = f(\F x) \\
  & = \sum_{i=1}^{k-1}{\delta_ix_i^2 + x_k^2h_{kk}(\F x)} 
    + 2x_k \sum_{j=k+1}^{n}{x_jh_{jk}(\F x)} 
    + \sum_{i=k+1}^{n}{\sum_{j=k+1}^{n}{x_ix_jh_{ij}(\F x)}} \\
  & = \sum_{i=1}^{k-1}{\delta_ix_i^2} 
    + h_{kk}(\F x)\Biggl(x_k + \sum_{j=k+1}^{n}{x_j\frac{h_{jk}(\F x)}{h_{kk}(\F x)}}\Biggr)^2 \\
  & \quad - h_{kk}(\F x)\Biggl(\sum_{j=k+1}^{n}{x_j\frac{h_{jk}(\F x)}{h_{kk}(\F x)}}\Biggr)^2 
    + \sum_{i=k+1}^{n}{\sum_{j=k+1}^{n}{x_ix_jh_{ij}(\F x)}} \\
  & = \sum_{i=1}^{k }{\delta_iy_i^2} + \sum_{i=k+1}^{n}{\sum_{j=k+1}^{n}{y_iy_j\tilde h_{ij}(\F x)}} \\
  & = \sum_{i=1}^{k }{\delta_iy_i^2} + \sum_{i=k+1}^{n}{\sum_{j=k+1}^{n}{y_iy_j\tilde h_{ij}\circ\Psi^{-1}(\F y)}}
\end{align*}
where $\tilde{h}_{ij}\in C^\infty(W, \RR)$. This completes the induction step.
\end{proof}

We point out that with the assumptions of Theorem \ref{theorem:12-6} $p$ is the only critical
point in $U$. If $M$ is compact and $f\in C^\infty(W, \RR)$ is a Morse function then $f$ has
only finitely many critical points. Among them there will always be at least one
local minimum (index $\lambda = 0$) and at least one local maximum (index $\lambda=n$).

\begin{definition}\label{definition:12-7}
  Let $f\in C^\infty(W, \RR)$ be a Morse function. A smooth tangent
vector field $X$ on $M$ is said to be \Index{gradient-like} for $f$, if the following conditions
are satisfied:
\begin{enumerate}[(i)]
  \item For every point $p\in M, \dd_pf(X(p)) > 0$.
  \item If $p\in M^n$ is a critical point of $f$ then there exists a $C^\infty$-chart $h:U\to h(U)\subseteq\RR^n$
  with $p\in U$ and $h(p) = 0$ such that
  \[
    f\circ h^{-1}(\F x) = f(p) - x_1^2 - \cdots - x_\lambda^2 + x_{\lambda+1}^2 + \cdots + x_n^2, \qquad \F x\in h(U).
  \]
  and $h_*X_{|U} = \grad(f\circ h^{-1})$.
\end{enumerate}
\end{definition}

A smooth parametrized curve $\alpha:I\to M$ is an \Index{integral curve} for $X$, if
\[
  \alpha'(t) = X(\alpha(t)),\qquad \text{ for } t\in I.
\]

Hence one gets $(f\circ\alpha)'(t) = \dd_{\alpha(t)}f(X(\alpha(t)))$. If $\alpha(I)$ does not contain any
critical points, then $f\circ\alpha:I\to\RR$ is a monotone increasing function by condition (i).

\begin{lemma}\label{lemma:12-8}
  Every Morse function on $M$ admits a gradient-like vector field.
\end{lemma}


\begin{proof}
We can find a $C^\infty$-atlas $(U_\alpha, h_\alpha)_{\alpha\in A}$ for $M$ which satisfies the following two conditions:
\begin{enumerate}[(i)]
  \item Every critical point of $f$ belongs to just one of the coordinate patches $U_\alpha$.
  \item For any $\alpha\in A$ either $f$ has no critical point in $U_\alpha$ or $f$ has precisely
    one critical point $p$ in $U_\alpha, h_\alpha(p) = 0$, and $f\circ h_\alpha^{-1}$ has the form 
    listed in Example \ref{example:12-5}.
\end{enumerate}

Let $X_\alpha$ be a tangent vector field on $U_\alpha$ determined by $X_\alpha = (h^{-1}_\alpha)_*(\grad (f\circ h_\alpha^{-1}))$.
Choose a smooth partition of unity $(\rho_\alpha)_{\alpha\in A}$ subordinate to $(U_\alpha)_{\alpha\in A}$, and define a smooth 
tangent vector field on $M$ by
\[
  X = \sum_{\alpha\in A}^{}{\rho_\alpha X_\alpha},
\]
where $\rho_\alpha X_\alpha$ is taken to be 0 outside $U_\alpha$. If $p\in M$ is not a critical point for $f$ then, for 
every $\alpha\in A$ with $p\in U_\alpha$ and $q\in h_\alpha(p)$, we have 
\[
  \dd_pf(X_\alpha(p)) 
  = \dd_q(f\circ h_\alpha^{-1})(X_\alpha(p))(\grad (f\circ h_\alpha^{-1})) > 0
\]
Indeed, there is at least one $\alpha$ with $\rho_\alpha(p)>0$ and 
\[
  \dd_pf(X(p)) = \sum_{\alpha\in A}^{}\rho_\alpha(p)\dd_pf(X_\alpha(p)).
\]

We see that $X$ satisfies condition (i) in Definition \ref{definition:12-7}.

If $p$ is a critical point of $f$ then there exists a unique $\alpha\in A$ with $p\in U_\alpha$. It
follows from (a) that $X$ coincides with $X_\alpha$. on a neighborhood of $p$, and condition
(b) above shows that assertion (ii) in Definition \ref{definition:12-7} is satisfied.
\end{proof}

The next lemma relates the index of a Morse function to the local index of vector
fields as defined in Chapter 11.

\begin{lemma}\label{lemma:12-9}
  Let $f$ be a Morse function on $M$ and $X$ a smooth tangent vector
field such that $\dd_pf(X(p)) > 0$ for every $p\in M$ that is not a critical point for $f$.
Let $p_0\in M$ be a critical point for $f$ of index $\lambda$. If $X(p_0) = 0$, then
\[
  \iota(X; p_0) = (-1)^\lambda.
\]
\end{lemma}

\begin{proof}
  We choose a gradient-like vector field $\tilde X$. By Definition \ref{definition:12-7}.(ii) and
Example \ref{example:12-5},
\[
  \iota(\tilde{X}; p_0) = (-1)^\lambda.
\]
Let $U$ be an open neighborhood of $p_0$ that is diffeomorphic to $\RR^n$ and chosen so
small that $p_0$ is the only critical point in $U$. The inequalities
\[
  \dd_pf(X(p)) > 0,\qquad \dd_pf(\tilde{X}(p)) > 0
\]
valid for $p\in U-\{p_0\}$, show that $X(p)$ and $\tilde X(p)$ belongs to the same open half-space in $T_pM$. Thus 
\[
  (1-t)X(p) + t\tilde{X}(p) \qquad (0\le t\le 1)
\]
defines a homotopy between $X$ and $\title{X}$ considered as maps from $U-\{p_0\}$ to $\RR^n-\{0\}$, and 
$\iota(X; p_0) = \iota(\tilde{X}; p_0)$.
\end{proof}

\begin{remark}\label{remark:12-10}
  Given a Riemannian metric on $M$ and $f\in C^\infty(M, \RR)$, one can define the gradient vector field $\grad f$ by 
  the equation
  \[
    \langle \grad_p (f), X_p\rangle = \dd_pf(X_p)
  \]
  for all $X_p\in T_pM$. Then Lemma \ref{lemma:12-9} holds for $\grad(f)$.
\end{remark}

\begin{theorem}\label{theorem:12-11}
  Let $M^n$ be a compact differentiable manifold and $X$ a smooth
tangent vector field on $M^n$ with isolated singularities. Let $f\in C^\infty(M, \RR)$ be a
Morse function and $c_\lambda$ the number of critical points of index $\lambda$ for $f$. Then we
have that
\[
  \R{Index}(X) = \sum_{\lambda = 0}^{n }{(-1)^\lambda c_\lambda}.
\]
\end{theorem}

\begin{proof}
  It is a consequence of Theorem \ref{theorem:11-27} that any two tangent vector fields
with isolated singularities have the same index. Thus we may assume that $X$ is
gradient-like for $f$. The zeros for $X$ are exactly the critical points of $f$, and the
claimed formula follows from Lemma \ref{lemma:12-9}.

It is a consequence of the above theorem that the sum
\begin{align}\label{eq:12-13}
  \sum_{\lambda = 0}^{n }{(-1)^\lambda c_\lambda}
\end{align}
is independent of the choice of Morse function $f\in C^\infty(M, \RR)$. Given Theorem \ref{theorem:12-11}, the 
Poincare-Hopf theorem is the statement that the sum \eqref{eq:12-13} is equal to the Euler characteristic; cf. \eqref{eq:12-2}.

We will give a proof of this based on the two lemmas below, whose proofs in
turn involve methods from dynamical systems and ordinary differential equations,
and will be postponed to Appendix C.

Let us fix a compact manifold $M^n$ and a Morse function $f$ on $M$. For $a\in\RR$ we set
\[
  M(a) = \{p\in M\mid f(p)<a\}.
\]

Recall that a number $a\in\RR$ is a \Index{critical value} if $f^{-1}(a)$ contains at least one
critical point.

\begin{lemma}\label{lemma:12-12}
  If there are no critical values in the interval $[a_1, a_2]$, then $M(a_1)$ and $M(a_2)$ are diffeomorphic.
\end{lemma}

\begin{lemma}\label{lemma:12-13}
  Suppose that $a$ is a critical value and that $p_1, \ldots, p_r$ are the critical
points in $f^{-1}(a)$. Let $p_i$ have index $\lambda_i$. There exists an $\epsilon > 0$, and disjoint open
neighbourhoods $U_i$ of $p_i$, such that
\begin{enumerate}[(i)]
  \item $p_1,\ldots,p_r$ are the only critical points in $f^{-1}([a-\epsilon, a+\epsilon])$.
  \item $U_i$ is diffeomorphic to an open contractible subset of $\RR^n$.
  \item $U_i\cap M(a-\epsilon)$ is diffeomorphic to $S^{\lambda_i-1}\times V_i$, where $V_i$ is an open
    contractible subsets of $\RR^{n-\lambda_i+1}$ (in particular $U_i\cap M(a-\epsilon) = \ns$ if $\lambda_i=0$).
  \item $M(a+\epsilon)$ is diffeomorphic to $U_1\cup\ldots\cup U_r\cup M(a-\epsilon)$.
\end{enumerate}
\end{lemma}
\end{proof}

\begin{proposition}\label{proposition:12-14}
  In the situation of Lemma \ref{lemma:12-13} suppose that $M(a-\epsilon)$ has
  finite-dimensional cohomology. Then the same will be true for $M(a + \epsilon)$, and
  \[
    \chi(M(a+\epsilon)) = \chi(M(a-\epsilon)) + \sum_{i=1}^{r}{(-1)^{\lambda_i}}.
  \]
\end{proposition}

\begin{proof}
  For $U=U_1\cup\ldots\cup U_r$, Lemma \ref{lemma:12-13} and Corollary \ref{corollary:6-10} imply that 
  \begin{align*}
    H^p(U) \se \left\{\begin{aligned}
      &0 && \text{ if } p\neq 0 \\
      &\RR^r && \text{ if } p = 0
    \end{aligned}\right.
  \end{align*}
  This gives $\chi(U) = r$. Condition (iii) of Lemma \ref{lemma:12-13} shows that $U_i\cap M(a-\epsilon)$
  is homotopy equivalent to $S^{\lambda_i-1}$, and Example \ref{example:9-29} gives
  \[
    \chi(U_i\cap M(a-\epsilon)) 
    = 1 + (-1)^{\lambda_i - 1}
  \]
  Since $U\cap M(a-\epsilon)$ is a disjoint union of the sets $U_i\cap M(a-\epsilon)$, it has a
  finite-dimensional de Rham cohomology, and 
  \[
    \chi(U\cap M(a-\epsilon)) 
    = \sum_{i=1 }^{r }{(1+(-1)^{\lambda_i-1})}
    = r - \sum_{i=1 }^{r }{(-1)^{\lambda_i}}
    = \chi(U) - \sum_{i=1 }^{r }{(-1)^{\lambda_i}}.
  \]
  The claimed formula now follows from Lemma \ref{lemma:12-13}.(iv) and the lemma below,
  applied to $U$ and $V = M(a - \epsilon)$.
\end{proof}

\begin{lemma}\label{lemma:12-15}
  Let $U$ and $V$ be open subsets of a smooth manifold. If $U, V$ and
$U\cap V$ have finite dimensional de Rham cohomology, the same is true for $U\cup V$, and
\[
  \chi(U\cup V) = \chi(U) + \chi(V) - \chi(U\cap V).
\]
\end{lemma}

\begin{proof}
  We use the long exact Mayer-Vietoris sequence
  \[
    \cdots\ra H^{p-1}(U\cap V)\ra H^p(U\cup V)\ra H^p(U)\oplus H^p(V)\ra H^p(U\cap V)\ra\cdots
  \]
  First we conclude that $\dim H^P(U\cup V)<\infty$. Second, the altematidg sum of the
  dimensions of the vector spaces in an exact sequence is equal to zero; cf. Exercise \ref{exercise:4-4}.
\end{proof}

\begin{theorem}\label{theorem:12-16}
  If $f$ is a Morse function on the compact manifold $M^n$, then
  \[
    \chi(M^n) = \sum_{\lambda=0}^{n }{(-1)^\lambda c_\lambda},
  \]
  where $c_\lambda$ denotes the number of critical points for $f$ of index $\lambda$.
\end{theorem}

\begin{proof}
  Let $a_1 < a_2 < \ldots < a_{k-1} < a_k$ be the critical values. Choose real
  numbers $b_0 < a_1. b_j\in (a_j, a_{j+1})$ for $1\le j\le k-1$ and $b_k > a_k$. 
  Lemma \ref{lemma:12-12} shows that the dimensions of $H^d(M(b_j))$ are independent of 
  the choice of $b_j$ from the relevant interval. If $M(b_{j-1})$ has finite-dimensional de 
  Rham cohomology, the same will be true for $M(b_j)$ according to Proposition \ref{proposition:12-14}, 
  and
  \begin{align}\label{eq:12-14}
    \chi(M(b_j)) - \chi(M(b_{j-1})) = \sum_{p\in f^{-1}(a_j)}^{}{(-1)^\lambda(p)}
  \end{align}
  Here the sum runs over the critical points $p\in f^{-1}(a_j)$, and $\lambda(p)$ 
  denotes the index of $p$. We can start from $M(b_0) = 0$. An induction argument shows that
  $\dim H^d(M(b_j))<\infty$ for all $j$ and $d$. The sum of the formulas of \eqref{eq:12-14} for
  $1\le j\le k$ gives
  \[
    \chi(M) = \chi(M(b_k))
    = \sum_{p }^{}{(-1)^{\lambda(p)}}
  \]
  where $p$ runs over the critical points.
\end{proof}

The Poincare-Hopf theorem \ref{theorem:12-1} follows by combining Theorems \ref{theorem:12-11} 
and \ref{theorem:12-16}.


\begin{corollary}\label{corollary:12-17}
  If $M^n$ is compact and of odd dimension $n$ then $\chi(M^n) = 0$.
\end{corollary}

\begin{proof}
  Let $f$ be a Morse function on $M$. Then $-f$ is also a Morse function,
  and $-f$ has the same critical points as $f$. If a critical point $p$ has index $\lambda$ with
  respect to $f$, then $p$ has index $n-\lambda$ with respect to $-f$. Theorem \ref{theorem:12-16} applied
  to both $f$ and $-f$ gives
  \[
    \chi(M) 
    = \sum_{\lambda=0}^{n }{(-1)^\lambda c_\lambda}
    = \sum_{\lambda=0}^{n }{(-1)^{n-\lambda}c_{\lambda}}.
  \]
  The two sums differ by the factor $(-1)^n$, and the assertion follows.
\end{proof}


\begin{example}[Gauss-Bonnet in $\RR^3$]\label{example:12-18}
  We consider a compact regular surface $S\subseteq\RR^3$, oriented by means of the Gauss 
  map $N: S\to S^2$. The Gauss curvature of $S$ at the point $p$ is
  \[
    K(p) = \det(\!\dd_pN);\qquad T_pS = \{p\}^\perp = T_pS^2.
  \]
  Sard's theorem implies that we can find a pair of antipodal points in $S^2$ that are
  both regular values of $N$. After a suitable rotation of the entire situation we can
  assume that $p_\pm = (0, 0, \pm1)$ are regular values of $N$.

  Let $f\in C^\infty(S, \RR)$ be the projection on the third coordinate axis of $\RR^3$. The
  critical points $p\in S$ of $f$ are exactly the points for which $T_pS$ is parallel with
  the $x_1, x_2$-plane, i.e. $N(p) = p_\pm$. At such a point $p$, the differential of the Gauss
  map $\dd_pN$ is an isomorphism. Hence $K(p)\neq 0$. A neighborhood of $p$ in $S$ can be
  parametrized by $(u, v, f (u, v))$, and in these local coordinates the Gauss curvature
  has the following expression:
  \[
    K = \left(1+\left(\frac{\partial f }{\partial u }\right)^2 + \left(\frac{\partial f }{\partial v }\right)^2\right)^{-1}
    \det\begin{pmatrix}
      \frac{\partial^2 f}{\partial u^2} & \frac{\partial^2 f}{\partial u\partial v}\\[.75em]
      \frac{\partial^2 f}{\partial v\partial u} & \frac{\partial^2 f}{\partial v^2}
    \end{pmatrix}
  \]
  (see e.g. [do Carmo] page 163). Since $K(p)\neq 0$, the determinant in the expression
  does not vanish at $p$, so $p$ is a non-degenerate critical point for $f$. If $K(p) > 0$
  the determinant is positive and $p$ has index 0 or 2. If $K(p) < 0$ the determinant
  is negative and $p$ has index 1. We apply Theorem \ref{theorem:12-16} to get
  \[
    \chi(S) = \#\{p\in S \mid N(p)=p_\pm, K(p)>0\} 
    - \#\{p\in S \mid N(p)=p_\pm, K(p)<0\}.
  \]
  Since $p_+$ is regular value for $N$, we have by Theorem \ref{theorem:11-9}
  \[
    \R{deg}{N} = \#\{p\in N^{-1}(p_+)\mid K(p)>0\}
    - \#\{p\in N^{-1}(p_+)\mid K(p)<0\}.
  \]
  and analogously with $p_{-}$ instead of $p_+$. It follows that
  \begin{align}\label{eq:12-15}
    \chi(S) = 2\;\R{deg}(N).
  \end{align}
  The map 
  \[
    \alt^2(\!\dd N_p):\alt^2(T_{N_p}S^2) 
    = \alt^2(T_pS)\to\alt^2(T_p(S))
  \]
  is multiplication by $\det(\!\dd_pN) = K(p)$, so $N^*(\R{vol}_{S^2})=K(p)\R{vol}_S$. Hence 
  \[
    \int_S K\R{vol}_S 
    = \int_{S^2} N^*(\R{vol}_{S^2})
    = (\R{deg}N)\int_{S^2}\R{vol}_{S^2}
    = 4\pi \R{deg}N.
  \]
  Combined with \eqref{eq:12-15} this yields the \Index{Gauss-Bonnet formula}
  \begin{align}\label{eq:12-16}
    \frac1{2\pi}\int_S K\R{vol}_S = \chi(S).
  \end{align}
\end{example}

\begin{example}\label{example:12-19}
  Consider the torus $T$ in $\RR^3$ . The height function $f:T\to\RR$ is a
  Morse function with the four indicated critical points. The Gauss curvature of $T$ is
  \begin{figure}[!htb]
    \centering
    \includegraphics[width=.5\linewidth]{./pics/chap12-1-o.pdf}
    \label{fig:12-1}
  \end{figure}
  positive at $p$ and $s$, but negative at $q$ and $r$ (cf. Example \ref{example:12-18}). 
  Hence $f$ is a Morse function on $T$. The index at $p, q, r$ and $s$ is 0, 1, 1 and 2, respectively. 
  Theorem \ref{theorem:12-16} gives $\chi(T) = 0$. Since we know that $\dim H^0(T) = \dim H^2(T) = 1$, we
  can calculate $\dim H^1(T) = 2$.
\end{example}

\begin{example}[Morse function on $\B{RP}^n$]\label{example:12-20}
  Real functions on $\B{RP}^n$ are equivalent to even functions $f:S^n\to\RR$, 
  i.e. $f(x) = f(-x)$ for all $x\in S^n$. Let us try
  \[
    f(\F x) = \sum_{i=1 }^{n }{a_ix_i^2}
  \] 
  for $\F x = (x_0, x_1, \ldots, x_n)\in S^n\subseteq\RR^{n+1}$, and real numbers $a_i$. The differential 
  of $f$ at $\F x$ is given by  
  \[
    \dd_xf(v_0, \ldots, v_n) = 2\sum_{i=1 }^{n }{a_ix_iv_i},
  \]  
  where $u=(v_0, v_1, \ldots, v_n)\in T_xS^n$ so that 
  \[
    \sum_{i=1 }^{n }{x_iv_i} = 0.
  \]
  Thus $\F x$ is a critical point for $f$ if and only if the vectors $\F x$ and
  $(a_0x_0, a_1x_1, \ldots, a_nx_n)$ are linearly independent. If the coefficients $a_i$ are
  distinct, this occurs precisely for $\F x = \pm e_i = (0,\ldots, \pm1,\ldots,0)$, and $f$ has exactly $2n+2$ critical 
  points. The induced smooth map $i:\B{RP}^n\to\RR$ then has $n+1$ critical points $[e_i]$. In a neighborhood 
  of $\pm e_0\in S^n$ we have the charts $h$ with
  \[
    h_\pm^{-1}(u_1, \ldots, u_n)
    = \left(\pm\sqrt{1-\sum\nolimits_{i=1}^nu_i^2},\; u_1, \ldots, u_n\right),
  \]
  and in a neighborhood of $0\in\RR^n$,
  \[
    f\circ h_\pm^{-1}(u_1, \ldots, u_n)
    = a_0\biggl(1-\sum_{i=1}^n u_i^2\biggr) + \sum_{i=1}^{n }{a_iu_i^2}
    = a_0 + \sum_{i=1}^{n }{(a_i-a_0)u_i^2}.
  \]
  The matrix of the second-order partial derivatives for $f\circ h_\pm^{-1}$ (at 0) is the 
  diagonal matrix
  \[
    \R{diag}(2(a_1-a_0), 2(a_2-a_0), \ldots, 2(a_n-a_0)).
  \]

  Hence $\pm e_0$ are non-degenerate critical points for $f$; the index for each is equal
  to the number of indices $i$ with $1\le i\le n$ and $a_i < a_0$. An analogous result
  holds for the other critical points $\pm e_j$. For simplicity, suppose that 
  $a_0 < a_1 <a_2 < \ldots < a_n$. Then the two critical points $\pm e_j$ for $f:S^n\to\RR$ have index $j$.
  The induced function $\tilde{f}:\B{RP}^n\to\RR$ is a Morse function with critical points $[e_j]$ of
  index $j$. We apply Theorem \ref{theorem:12-16} to $\tilde{f}$. Since $c_\lambda=1$ for $0\le\lambda\le n$, 
  we get
  \begin{align*}
    \chi(\B{RP}^n)
    = \left\{\begin{aligned}
      1\quad & \text{ if $n$ is even } \\
      0\quad & \text{ if $n$ is odd }. 
    \end{aligned}\right.
  \end{align*}
  This agrees with Example \ref{example:9-31}.
\end{example}

