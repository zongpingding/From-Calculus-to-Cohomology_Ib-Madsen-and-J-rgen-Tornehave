\chapter{Fiber Bundles and Vector Bundle}
\begin{definition}\label{definition:15-1}
  A \Index{fiber bundle} consists of three topological spaces $E, B, F$ and a continuous 
  map $\pi:E\to B$, such that the following condition is satisfied: Each
  $b\in B$ has an open neighborhood $U_b$ and a homeomorphism
  \[
    h:U_n\times F\to \pi^{-1}(U_b)
  \]
  such that $\pi\circ h=\R{proj}_1$.
\end{definition}

The space $E$ is called the \Index{total space}, $B$ the \Index{base space} and $F$ the (typical) fiber.
The pre-image $\pi^{-1}(x)$, frequently denoted by $F_x$ , is called the fiber over $x$. A
fiber bundle is said to be smooth, if $E, B$ and $F$ are smooth manifolds, $\pi$ is a
smooth map and the $h$ above can be chosen to be diffeomorphisms. One may
think of a fiber bundle as a continuous (smooth) family of topological spaces $F_x$
(all of them homeomorphic to $F$), indexed by $x\in B$.


The most obvious example is the \Index{product fiber} bundle $\varepsilon_B^F = (B\times F, B, F, \R{proj}_1)$.
In general, the condition of Definition \ref{definition:15-1} expresses that the ``family'' is locally
trivial.


\begin{example}[The canonical line bundle]\label{example:15-2}\index{canonical line bundle}
  At the beginning of Chapter 14 we considered the action of $S^1$ on $S^{2n+1}$ with orbit space $\CP^n$. 
  We view this as an action from the right, $\F z\cdot\lambda=(z_0\lambda,\ldots, z_n\lambda)$. The circle acts 
  also on $S^{2n+1}\times\CC^k, (\F z, \F u)\lambda = (\F z\lambda, \lambda^{-1}\F u)$. The associated orbit 
  space is denoted $S^{2n+1}\times_{S^1}\CC^k$. The projection on the first factor gives a continuous map
  \[
    \pi:S^{2n+1}\times_{S^1}\CC^k\to \CP^n
  \]
  with fiber $\CC^k$ Similarly, if $S^1$ acts continuously (from the left) on any topological
  space $F$ we get
  \[
    \pi:S^{2n+1}\times_{S^1}F\to \CP^n.
  \]
  This is a fiber bundle with fiber $F$. Indeed, we have the open sets $U_j$ displayed
  at the beginning of Chapter 14 which cover $\CP^n$, and the smooth sections
  \[
    s_j:U_j\to S^{2n+1}
  \]
  from (14.\ref{eq:14-7}). We can define local trivializations by
  \[
    \hat{s}_j([\F z],\F u) = (s_j[\F z], \F u)\in\pi^{-1}(U_j)
  \]
  for $[\F z]\in U_j$ and $\F u\in F$. If $F$ is a smooth manifold with smooth $S^1$-action then
  we obtain a \Index{smooth fiber bundle}.

  If we take $F = \CC$ with its usual action of $S^1$ then we obtain the \Index{dual Hopf bundle},
  or canonical line bundle, over $\CP^n$. It will be denoted
  \[
    H_n = S^{2n+1}\times_{S^1}\CC
  \]
  It is a vector bundle in the sense of Definition \ref{definition:15-4} below.
\end{example}

\begin{example}\label{example:15-3}
  Over the real projective space $\B{RP}^n$ we have similar bundles
  $S^n\times_{S^0} F$, where $S^0 = \{\pm1\}$. In particular $D(H) = S^1\times_{s^0} D^1$ is 
  the Mobius band.
\end{example}

\begin{definition}\label{definition:15-4}
  A \Index{vector bundle} $\xi = (E, B, V, \pi)$ is a fiber bundle where the
  \Index{typical fiber} $V$ and each $\pi^{-1}(x)$ are vector spaces, and where the local 
  homeomorphism $h:U_b\times V\to\pi^{-1}(U_b)$ can be chosen so that $h(x, -):V\to\pi^{-1}(x)$ is
  a linear isomorphism for each $x\in U_b$.
\end{definition}

Vector bundles can be real, complex or quaternion depending on which category
$V$ and $h(x, -)$ in Definition \ref{definition:15-4} belong to. For the time being we concentrate
on real vector bundles. A smooth vector bundle is a vector bundle that is also a
smooth fiber bundle. The dimension of a vector bundle is the dimension of the
fiber. Vector bundles of dimension 1 are called line bundles.


We mostly denote vector bundles by small Greek letters. If $\xi$ is a vector bundle
then $E(\xi)$ will denote its total space and $F_b(\xi)$, or just $\xi_b$, its fiber over $b$. If
$W\subset B$ then we write $\xi_{|W}$ for the restriction of $\xi$ to $W$, i.e. $E(\xi_{|W})=\pi^{-1}_\xi(W)$.

\begin{example}[The tangent bundle]\label{example:15-5}\index{tangent bundle}
  Let $M^n\subset\RR^{n+k}$ be a smooth manifold. Consider 
  \[
    TM = \{(p, v)\in M\times\RR^{n+k} \mid v\in T_pM\},\qquad \pi(p, v) = p.
  \]
  The fiber over $p\in M$ is the tangent space $T_pM$. We show that the triple
  $\tau_M = (TM, M, \pi)$ is a vector bundle. Let $b\in M$. Choose a parametrization
  $(V, g)$ around $b,g:W\to U,\quad W\subseteq\RR^n$ and let
  \[
    h:U\times\RR^n\to\pi^{-1}(U);\quad h(x, v) = D_{g_{g^{-1}(x)}}(v).
  \]
  This gives the required local triviality.
\end{example}

\begin{example}[The normal bundle]\label{example:15-6}\index{normal bundle}
  Let $M^n\subset\RR^{n+k}$ be a smooth manifold. Let 
  \[
    N_p(M) = (T_pM)^\perp
  \]
  be the orthogonal complement to $T_pM\subset\RR^{n+k}$. Set 
  \[
    F(\nu_M) = \bigcup_{p\in M} N_p(M)\subset M\times\RR^{n+k}; \qquad \pi(v) = p\text{ when } v\in N_p(M).
  \]
  We must show that $\nu_M$ is locally trivial. For each $p_0\in M$, the proof of Lemma \ref{lemma:9-21} 
  produced vector fields
  \[
    X_1, \ldots, X_n, Y_1, \ldots, Y_k:W\to\RR^{n+k}
  \]
  defined in a neighborhood $W$ of $p_0$, with the property that they are orthonormal for
  each $p\in W$ and such that $X_1(p), \ldots, X_n(p)\in T_pM$. Hence $Y_1(p), \ldots, Y_k(p)\in N_p(M)$.
  is a basis and the map
  \[
    h:W\times\RR^k\to\pi^{-1}(W);\qquad h(p, \F t) = \sum_{i=1 }^{k}{t_iY_i(p)}
  \]
  is a local trivialization.
\end{example}

\begin{definition}\label{definition:15-7}
  \begin{enumerate}
    \item A map $(f, \hat f)$ between (smooth) fiber bundles $(E, B, \pi)$ and $(E', B', \pi')$
      is a pair of (smooth) continuous maps
      \[
        f:B\to B',\quad \hat f:E\to E'
      \]
      such that $\pi'\circ \hat f=f\circ \pi$.
    \item A homomorphism between (smooth) vector bundles $\xi$ and $\xi'$ is a (smooth)
      fiber bundle map such that $\hat f:\pi^{-1}(x)\to(\pi')^{-1}(f(x))$ is linear for all 
      $x\in B$.
  \end{enumerate}
\end{definition}

\begin{example}\label{example:15-8}
  A smooth map $f: M\to M'$ between smooth manifolds induces a map of tangent bundles $(f, Tf)$, where
  \[
    T_pf = D_pf:T_pM\to T_pM'
  \]
  is the derivative of $f$.
\end{example}

\begin{definition}\label{definition:15-9}
  Vector bundles $\xi$ and $\eta$ over the same base space $B$ are called
  isomorphic, if there exist homomorphisms $(\id_B, \hat f)$ and $(\id_B, \hat g)$ between them
  such that $\hat f\circ\hat g=\id = \hat g\circ\hat f$. A vector bundle which is isomorphic to a product
  bundle is called trivial, and a specific isomorphism is called a \Index{trivialization}.
\end{definition}

In the above definition the homomorphisms $\hat f$ and $\hat g$ are assumed to be smooth
when the vector bundles are smooth. The next lemma is a convenient tool for
deciding if two bundles are isomorphic.

\begin{lemma}\label{lemma:15-10}
  A (smooth) continuous map $\hat f:E(\xi)\to E(\eta)$ of (smooth) vector
  bundles over $B$, which map the fiber $F_b(\xi)$ isomorphically onto the fiber $F_b(\eta)$, is
  a (smooth) \textit{isomorphism}\index{vector bundle!isomorphism}.
\end{lemma}

\begin{proof}
  Since $\hat f$ is a bijection, it is sufficient to show that $\hat f^{-1}$ is a (smooth)
  homomorphism of vector b~ndles (over $\id_B$) We need to check that $\hat f^{-1}$ is
  continuous (smooth). Since $\hat f$ is a fiberwise isomorphism, it is enough to examine
  \[
    \hat f^{-1}:\pi^{-1}_\eta(U)\to \pi^{-1}_\xi(U)
  \]
  where $\xi$ and $\eta$ are trival over $U$. Let 
  \[
    h:U\times\RR^n\to\pi^{-1}_\xi(U);\quad k:U\times\RR^n\to\pi^{-1}_\eta(U)
  \]
  be isomorphisms. Then 
  \[
    F = k\circ \hat f\circ h^{-1}:U\times\RR^n\to U\times\RR^n
  \]
  is an isomorphism of trival bundles and it has the from 
  \[
    F(x, v) = (x, F_2(x, v)), \quad x\in U.
  \]
  The map $x\to F_2(x, -)$ defines a map 
  \[
    \R{ad}(F_2):U\to\R{GL}_n(\RR).
  \]
  Conversely such a map induces a homomorphism $\hat f^{-1}:\pi^{-1}_\xi(U)\to\pi^{-1}_\eta(U)$. Note that 
  \[
    \R{ad}(F_2)^{-1}:U\to\R{GL}_n(\RR)
  \]
  determines $F^{-1}$. Finally, it is easy to see that $f$ is continuous (smooth) if and only
  if $\R{ad}(F_2)$ is. The lemma now follows because matrix inversion $(-)^{-1}: \R{GL}_n(\RR)\to\R{GL}_n(\RR)$
  a smooth map.
\end{proof}

\begin{definition}\label{definition:15-11}
  The direct sum $\xi\oplus\eta$ of two vector bundles over the same base space $B$ is the vector 
  bundle over $B$ with total space
  \[
    E(\xi\oplus\eta) = \{(v, w)\in E(\xi)\times E(\eta)\mid \pi_\xi(v)=\pi_\eta(w)\}
  \]
  and projection $\pi_{\pi\oplus\eta}(v, w)=\pi_\xi(v)=\pi_\eta(w)$. The fiber $(\xi\oplus\eta)_b$
  is equal to $\xi_b\oplus\eta_b$.
\end{definition}

\begin{definition}\label{definition:15-12}
  An \Index{inner product} on a (smooth) vector bundle $\xi$ is a (smooth)
  map $\phi:E(\xi\oplus\xi)\to\RR$ such that $\phi:F_b(\xi)\oplus F_b(\xi)\to\RR$ is an inner product on
  each fiber $F_b(\xi)$.
\end{definition}

\begin{proposition}\label{proposition:15-13}
  Every vector bundle over a compact $B$ has an inner product.
\end{proposition}

\begin{proof}
  Choose local trivializations
  \[
    h_i:U_i\times\RR^n\to\pi^{-1}_\xi(U_i)
  \]
  where $U_1,\ldots, U_r$ cover $B$, and choose a partition of unity $\{\alpha_i\}_{i=1}^r$
  with $\supp(\alpha_i)\subset U_i$.

  The usual inner product in $\RR^n$ induces an inner product in $U_i\times\RR^n$ and hence,
  via $h_i$, an inner product $\phi_i:\pi^{-1}_\xi(U_i)$. Now
  \[
    \phi(v, w) = \sum \phi_i(v, w)\alpha_i(\pi(v))
  \]
  is an inner product in $\xi$.
\end{proof}

An inner product in the tangent bundle $\tau_M$ of a smooth manifold is the same as
a \Index{Riemannian metric} on $M$ (cf. Definition \ref{definition:9-15}).

\begin{remark}\label{remark:15-14}
  In the above proposition we used the existence of a continuous
  partition of unity, i.e. continuous functions $\alpha_i:B\to [0, 1]$ with $\supp(\alpha_i)\subset U_i$
  and $\sum\alpha_i(b)=1$ for all $b\in B$. When $B$ is a smooth manifold the existence
  of a smooth partition of unity is proved in Appendix A. More generally, a
  Hausdorff space $B$ is called \Index{paracompact} if every open covering $\{U_\alpha\}$ has an
  open refinement $\{V_\beta\}$ which is locally finite. For a given open cover $\{U_\alpha\}$ of
  a paracompact space there exists a partition of unity subordinated to $\{U_\alpha\}$, i.e.
  continuous functions $s_\alpha:B\to [0,1]$ with $\supp(s_\alpha)\subset U_\alpha$ and such that each 
  $b\in B$ has an open neighborhood $V_b$ for which $\{\alpha\mid s_{\alpha|V_b}\neq 0\}$ is a finite set.
\end{remark}

\begin{definition}\label{definition:15-15}
  A (smooth) \Index{section} in a (smooth) fiber bundle $(E, B, F;\pi)$ is a (smooth) map $s:B\to E$ 
  such that $\pi\circ s = E$.
\end{definition}

The set of sections of a vector bundle $\xi$ is a vector space $\Gamma(\xi)$. One adds sections
by using the vector space structure of each fiber. The origin in $\Gamma(\xi)$ is the \Index{zero section}
which to $b\in B$ assigns the origin in the fiber $\xi_b$. If $\xi$ is a smooth vector
bundle then we let $\Omega^0(\xi)\subset\Gamma(\xi)$ denote the subspace of smooth sections.


It follows from local triviality that in a neighbourhood $U$ of each point of
the base, we can find sections $s_1,\ldots,s_n\in\Gamma(\xi_{|U})$ (or in $\Omega^0(\xi_{|U})$) 
such that $\{s_1(x), \ldots, s_n(x)\}$ is a basis of $\xi_x$. We call this a frame. If $\xi$ 
has an inner product we may even choose sections locally so that $\{s_1(x), \ldots, s_n(x)\}$ is an
orthogonal basis (Gram-Schmidt). We say that $\{s_1, \ldots, s_n\}$ is an orthonormal
frame for $\xi$ over $U$.

Let $(f, \id_B)$ be a homomorphism from $\xi$ to $\eta$, and let $\{s_i\}, \{t_i\}$ be frames over
$U$. Then $\hat f_x:\xi_x\to \eta_x$ is represented by a matrix, and we obtain a map
\begin{align}\label{eq:15-1}
  \R{ad}(\hat f):U\to M_n(\RR)
\end{align}
depending on the given frames. In the smooth situation $\R{ad}(\hat f)$ is smooth. Note
that $\hat f_x:\xi_x\to \eta_x$ is an isomorphism if and only if $\R{ad}(\hat f_x)\in \R{GL}_n(\RR)$. 
If $\xi$ and $\eta$ have an inner product, and $\{s_i\}, \{t_i\}$ are orthonormal frames, then $\hat f_x$ 
is isometric precisely if $\R{ad}(\hat f_x)\in O_n$, the orthogonal subgroup of $\R{GL}_n(\RR)$.

\begin{lemma}\label{lemma:15-16}
  Let $\xi$ and $\eta$ be (smooth) vector bundles with inner product over
  the compact space $B$, and let $\hat f:\xi\to \eta$ be an isomorphism. Then there exists an
  $\epsilon > 0$ such that every homomorphism $g:\xi\to\eta$ that satisfies $\|\hat f_b-\hat g_b\|<\epsilon$ 
  for $b\in B$ is also an isomorphism.  
\end{lemma}

\begin{proof}
  If $\xi$ and $\eta$ are trivial, then after choice of frames, $\hat f$ and $\hat g$ are represented
  by maps $\R{ad}(\hat f):B\to \R{GL}_n(\RR)$ and $\R{ad}(\hat g):B\to M_n(\RR)$. Since $B$ is 
  compact and $\R{GL}_n(\RR)$ is open, some $\epsilon$-neighborhood of $\R{ad}(\hat f)(B)$ in $M_n(\RR)$ 
  is still contained in $\R{GL}_n(\RR)$. But then $\R{ad}(\hat g)(B)\subset \R{GL}_n(\RR)$ when $\hat g$ 
  satisfies the condition of the lemma. In general, we can cover $B$ with a finite number of compact 
  neighborhoods over which the bundles are trivial, and take the minimum of the resulting epsilons.
\end{proof}

Given two smooth vector bundles $\xi$ and $\eta$, one might wonder if there is any
essential distinction between the notions of continuous and smooth isomorphism.
The next result shows that this is not the case. Along the same lines one may ask
if each isomorphism class of continuous vector bundles over a compact manifold
contains a smooth representative. This is indeed the case (cf. Exercise \ref{exercise:15-8}).

\begin{lemma}\label{lemma:15-17}
  If two smooth vector bundles $\xi$ and $\eta$ over the compact manifold
  $B$ are isomorphic as continuous bundles, then they are smoothly isomorphic.
\end{lemma}

\begin{proof}
  We choose a cover $U^1,\ldots, U^r$ of $B$ and smooth local orthonormal frames
  $s^i = (s_1^i,\ldots, s_n^i)$ and $t^i = (t_1^i,\ldots,t_n^i)$ for $\xi$ and $\eta$, over $U^i$.

  A continuous isomorphism $\hat f:\xi\to \eta$ gives continuous maps
  \[
    \R{ad}(\hat f^i):U^i\to\R{GL}_n(\RR).
  \]
  Let $G^i:U^i\to\R{GL}_n(\RR)$ be a smooth $\epsilon$-approximation with $\|G^i(x)-\R{ad}(\hat f^i_x)\|<\epsilon$
  for $x\in U^i$. Construct a smooth homomorphism $\hat g^i:\pi^{-1}_\xi(U^i)\to \pi^{-1}_\eta(U^i)$ with 
  $\R{ad}(\hat g^i) = G^i$ by the formula 
  \[
    \hat g^i_b \left(\sum \lambda_k s^k(b)\right) 
    = \sum t^k(b)\cdot G^i_{k\nu}(b)\cdot\lambda_\nu.
  \]
  Then $\R{ad}(\hat g^i)=G^i$, and $\|\hat f_b-\hat g^i_b\|<\epsilon$ for $b\in U^i$. We can then use a smooth 
  partition of unity $\{\alpha_i\}$ with $\supp(\alpha_i)\subset U^i$ to define 
  \[
    \hat g:\xi\to\eta,\quad 
    \hat g_b = \sum_{i=1 }^{r }{\alpha_i(b)\hat g_b^i}.
  \] 
  Then $\|\hat f_b-\hat g_b\| = \|f_b-\Sigma\alpha_i(b)\hat g_b^i\|\le \Sigma\alpha_i(b)\|f_b-\hat g_b^i\|\le\Sigma\alpha_i(b)\epsilon=\epsilon$.
  With $\epsilon$ as in Lemma \ref{lemma:15-16}, $\hat g$ becomes an isomorphism on every fiber and hence a smooth 
  isomorphism by Lemma \ref{lemma:15-10}.
\end{proof}

In Examples \ref{example:15-5} and \ref{example:15-6} we constructed two vector bundles associated to a
submanifold $M^n\subset \RR^{n +k}$, namely the tangent bundle $\tau$ and the normal bundle
$\nu$. It is obvious that $\tau\oplus\nu$ is a trivial vector bundle. Indeed, by construction
$\tau_p\oplus\nu_p=\RR^{n+k}$ for every $p\in M$, and there is a globally defined frame for $\tau\oplus\nu$.
Hence $\tau\oplus\nu\simee \varepsilon^{n+k}$, where $\varepsilon^{n+k}$ is the \Index{trivial bundle} over $M$ of dimension
$n + k$. We now give the general construction of complements of vector bundles\index{vector bundle!complement}.

\begin{theorem}\label{theorem:15-18}
  Every vector bundle $\xi$ over a compact base space $B$ has a complement $\eta$, i.e. $\xi\oplus\eta\simee\varepsilon^N$ 
  (for a suitably large $N$).
\end{theorem}

\begin{proof}
  Choose an open cover $U^1, \ldots, U^r$ of $B$ of B admitting trivializations $h_i$ of
  $\xi_{|U^i}$ and let $\{\alpha_i\}$ be a partition of unity with $\supp(\alpha_i)\subset U^i$.
  Denote by $f^i$ the composite
  \[
    \pi^{-1}_\xi(U^i)\xra[h^{-1}_i] U^i\times\RR^n\xra[\R{proj}_2] \RR^n,
  \]
  and define 
  \begin{align}\label{eq:15-2}
    \begin{matrix}
      S:E(\xi)\to B\times\RR^{nr};\\
      S(v) = (\pi_\xi(v), \alpha_1(\pi_\xi(v))f^1(v), \ldots, \alpha_r(\pi_\xi(v))f^r(v)).
    \end{matrix}
  \end{align}
  This is a fiberwise map and gives a homomorphism $S:\xi\to\varepsilon^{nr}$ which is an
  inclusion on each fiber. We give $\RR^{nr}$ the usual inner product and let
  \[
    E(\eta) = \{(b, v)\mid v\in S(F_b(\xi))^\perp\}.
  \]
  It is easy to see that 
  \[
    \eta = (E(\eta), B, \RR^{nr-n}, \R{proj}_1)
  \]
  is a vector bundle (cf. Example \ref{example:15-6}) and by definition $\xi\oplus\eta = \varepsilon^{nr}$.
\end{proof}

If $E$ in Theorem \ref{theorem:15-18} is smooth then so is the constructed complement $\eta$, provided
$h_i$ and $\alpha_i$ are choosen smooth. The above proof uses that $B$ is compact to ensure
$r < \infty$. The theorem is not true without the compactness condition; see however
Exercise \ref{exercise:15-10} when $\xi$ a smooth vector bundle -- it is the finite-dimensionality
which counts.

Let $\R{Vect}_n(B)$ denote the isomorphism classes of vector bundles over $B$ of
dimension $n$. Direct sum induces a map
\[
  \R{Vect}_n(B)\times \R{Vect}_m(B)\xra[\oplus] \R{Vect}_{n+m}(B)
\]
such that 
\[
  \R{Vect}(B) = \coprod_{n=0}^\infty \R{Vect}_n(B)
\]
becomes an abelian semigroup. The zero dimensional bundle $\varepsilon^0=B\times\{0\}$ is the unit element.

To any abelian semigroup $(V, +)$ one can associate an abelian group $(K(V), +)$
defined as the formal differences $a - b$, or pairs $(a, b)$, subject to the relation
\[
  (a+x) - (b+x) = a-b
\]
where $x\in V$ is arbitrary. The construction has the universal property that
any homomorphism from $V$ to an abelian group A factors over $K(V)$, i.e. is
induced from a homomorphism from $K(V)$ to $A$. The construction $V\to K(V)$,
often called the \Index{Grothendieck construction}, corresponds to the way the integers
are constructed from the natural numbers, except that we do not demand that
cancellation ``$x+a=y+a\Rightarrow x=y$'' holds in $V$. When $B$ is compact, we define
\begin{align}\label{eq:15-3}
  KO(B) = K(\R{Vect}(B))
\end{align}

By Theorem \ref{theorem:15-18} every element of $KO(B)$ has the form $[\xi]-[\varepsilon^k]$, 
where $[\xi]$ denotes the isomorphism class of the vector bundle $[\xi]$. Indeed
\[
  [\xi_1] -[\xi_2]
  = ([\xi_1]+[\eta_2]) - ([\xi_2]+[\eta_2])
  = [\xi_1\oplus\eta_2] - [\eta_2]
  = [\xi] - [\varepsilon^k]
\]
if we choose $\eta_2$ to be a complement to $\xi_2$.

\begin{example}\label{example:15-19}
  The normal bundle to the unit sphere $S^2\subseteq \RR^3$ is trivial, since
  the outward directed unit normal vector defines a global frame. We also know
  that $\tau_{S^2}\oplus\nu_{S^2}=\varepsilon^3$, such that
  \[
    [\tau_{S^2}] + [\varepsilon^1] = [\varepsilon^3]
  \]
  in $\R{Vect}(S^2)$. However, $[\tau_{S^2}]\neq [\varepsilon^2]$ in $\R{Vect}(S^2)$. 
  Indeed, if $[\tau_{S^2}]$ were equal to $[\varepsilon^2]$, then there would exist a 
  section $s\in\Gamma(\varepsilon^2)$ with $s(x)\neq 0$ for all $x\in S^2$.
  However, Theorem \ref{theorem:7-3} implies that $\varepsilon^2$ does not have a 
  non-zero section. We see that cancellation does not hold in $\R{Vect}(S^2)$.
\end{example}

\begin{definition}\label{definition:15-20}
  Let $f: X\to B$ be a continuous (smooth) map and $\xi$ a (smooth) vector bundle over $B$. 
  The pre-image or pull-back $f^*(\xi)$ is the vector bundle over $X$ given by
  \[
    E(f^*(\xi)) = \{(x, v)\in X\times E(\xi)\mid f(x)=\pi_\xi(v)\}, \quad 
    \pi_{f^*(\xi)} = \R{proj}_1.
  \]
  We note the homomorphism $(f, \hat f):f^*(\xi)\to \xi$ given by $\hat f(x, v) = v$. It is
  obvious that the pull-backs of isomorphic bundles are isomorphic, so $f^*$ induces homomorphisms
  \[
    f^*:\R{Vect}(B)\to \R{Vect}(X) \enspace\text{ and }\enspace
    f^*:KO(B) \to KO(X).
  \]
  and $(g\circ f)^*=f^*\circ g^*, \id^*=\id$. Thus $\R{Vect}(B)$ and $KO(B)$ become contravariant functors.
\end{definition}

\begin{theorem}\label{theorem:15-21}
  If $f_0$ and $f_1$ are homotopic maps, then $f^*_0(\xi)$ and $f^*_1(\xi)$ are isomorphic.
\end{theorem}

\begin{proof}
  Let $F:X\times I\to B, I=[0, 1]$ be a homotopy between $f_0$ and $f_1$,
  $f_0(x) = F(x,0)$ and $f_1(x) = F(x,1)$. When $t\in I$ we get $[f^*_t(\xi)]\in\R{Vect}(X)$.
  It is sufficient to see that the function $t\to [f^*_t(\xi)]$ is locally constant, and thus
  constant.

  Fix $t$ consider the bundle
  \[
    \zeta = \R{proj}_1^*f^*_t(\xi) \quad \text{and}\quad 
    \eta = F^*(\xi)
  \]
  over $X\times I$. Since $F = f_t\circ \R{proj}_1$ on $X\times \{t\}, \zeta=\eta$ on $X\times \{t\}$. We can 
  choose a fiberwise isomorphism
  \[\begin{tikzcd}
    E(\zeta) \arrow[rr, "\hat h"] \arrow[dr, ""] & & E(\eta) \arrow[dl, ""] \\
    & X\times\{t\} &
  \end{tikzcd}\]
  The first step is to extend $\hat h$ to a homomorphism of vector bundles on $X\times [t-\epsilon, t+\epsilon]$
  for some $\epsilon > 0$. This can be done as follows. Since $X$ is compact, there
  exists a finite cover $U_1, \ldots, U_r$ of $X$ with $\epsilon_i > 0$ such that both $\zeta$ and $\eta$ 
  are trivial on $U_i\times [t - \epsilon_i, t + \epsilon_i]$. We can extend $\hat h$ to
  \[\begin{tikzcd}
    E(\zeta) \arrow[rr, "\hat h_i"] \arrow[dr, ""] & & E(\eta) \arrow[dl, ""] \\
    & U_i\times [t - \epsilon_i, t + \epsilon_i] &
  \end{tikzcd}\]
  Let $\alpha_1, \ldots, \alpha_r$ be a partition of unity on $X$ with $\supp(\alpha_i)\subset U_i$. We define 
  \[\begin{tikzcd}
    E(\zeta) \arrow[rr, "\hat k"] \arrow[dr, ""] & & E(\eta) \arrow[dl, ""] \\
    & X\times [t - \epsilon, t + \epsilon] &
  \end{tikzcd}\]
  where $\epsilon=\min(\epsilon_i)$ by setting
  \[
    \hat k(v) = \sum \alpha_i(\R{proj}_1\circ \pi_\zeta(v))\cdot \hat h_i(v).
  \]
  Since $\hat h_i(v) = \hat h(v)$ when $\pi_\zeta(v)\in X\times \{t\}$, and since $\sum \alpha_i(x)\equiv 1$,
  we have $\hat k(v)=\hat h(v)$ on $X\times \{t\}$. In particular $\hat k$ is an isomorphism on $X\times\{t\}$.

  We finally show that $\hat k$ is an isomorphism in a neighborhood $X\times [t-\epsilon_1, t+\epsilon_1]$
  of $X\times \{t\}$. Since $X$ is compact, it suffices to show that $\hat k$ is an isomorphism on a neighborhood 
  $V(x, t)$ of any point $(x, t)\in X\times \{t\}$. Let $\F e$ and $\F s$ be frames of $\zeta$
  and $\eta$ in a neighborhood $W$ of $(x, t)$, and $\R{ad}(\hat k):W\to M_n(\RR)$ the resulting map, cf. \eqref{eq:15-1}.
  Since $\R{GL}_n(\RR)\subset M_n(\RR)$ is open and $\R{ad}(\hat k)(x, t)\in \R{GL}_n(\RR)$, there exists a neighborhood
  $V(x, t)$ where $\R{ad}(\hat k)\in \R{GL}_n(\RR)$, and $\hat k$ is an isomorphism.
\end{proof}

The above theorem expresses that $\R{Vect}(X)$ (and hence also $KO(X)$) is a homotopy 
functor\index{homotopy!functor}: homotopic maps $f\sime g: X\to Y$ induce the same map
\[
  f^*=g^*:\R{Vect}(Y)\to \R{Vect}(X).
\]

\begin{corollary}\label{corollary:15-22}
  Every vector bundle over a contractible base space is trivial.
\end{corollary}

\begin{proof}
  With our assumption $\id_B \sime f$, where $f$ is the constant map with value $f(B)=\{b\}$.
  Hence $f^*(\xi)\simee \xi$. But $f^*(\xi)$ is trival by construction when $f$ is constant.
\end{proof}


In the above we have concentrated on \textit{real} vector bundles\index{vector bundle!real}. There is a completely
analogous notion of \textit{complex} (or even quaternion) vector bundles. In Definition
\ref{definition:15-4} one simply requires $V$ and $\pi^{-1}(x)$ to be complex vector spaces and $h(x, -)$
to be a complex isomorphism. The direct sum of complex vector bundles is a
complex vector bundle\index{vector bundle!complex}. A hermitian inner product on a complex vector bundle is
a map $\phi$ as in Definition \ref{definition:15-15} but such that it induces a hermitian inner product
in each fiber. Proposition \ref{proposition:15-13} and Theorem \ref{theorem:15-18} and \ref{theorem:15-21} have obvious
analogues for complex vector bundles.


The isomorphism class of complex vector bundles over $B$ of complex dimension $n$ is 
denoted $\R{Vect}_n^\CC(B)$. These sets give rise to a semigroup whose corresponding 
group (for compact $B$) is traditionally denoted
\[
  K(B) = K(\R{Vect}^\CC(B))
\]
It is a contravariant homotopy functor of $B$, often somewhat easier to calculate
than its real analogue $KO(B)$.