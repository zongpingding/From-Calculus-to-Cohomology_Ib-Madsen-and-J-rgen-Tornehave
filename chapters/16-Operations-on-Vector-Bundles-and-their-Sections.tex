\newcommand{\otr}[1][R]{\otimes_#1}%
\chapter{Operations on Vector Bundles and their Sections}
The main operations to be considered are tensor products and exterior products.
We begin with a description of these operations on vector spaces, then apply them
fiberwise to vector bundles, and end with the relation between the constructions
on bundles and their equivalent constructions on spaces of sections.


Let $R$ be a unital commutative ring and let $V$ and $W$ be $R$-modules. In the simplest
applications $R = \RR$ or $C$ and $V$ and $W$ are $R$-vector spaces, but we present the
definitions in the general setting. Denote by $R[V\times W]$ the free $R$-module with
basis the set $V\times W$, i.e. the space of maps from the set $V\times W$ to $R$ that are
zero except for a finite number of points in $V\times W$. In $R[V\times W]$, we consider
the submodule $R(V, W)$ which is generated (via finite linear combinations) by
elements of the form
\begin{align}\label{eq:16-1}
  \begin{aligned}
    & (v_1+v_2, w) - (v_1, w) - (v_2, w) \\
    & (v, w_1+w_2) - (v, w_1) - (v, w_2) \\
    & (rv, w) - r(v, w)\\
    & (v, rw) - r(v, w)
  \end{aligned}
\end{align}
where $v_i\in V, w_i\in W$ and $r\in R$.

\begin{definition}\label{definition:16-1}
  The \Index{tensor product} $V\otimes_{R} W$ of two $R$-modules is the quotient module $R[V\times W]/R(V, W)$.
\end{definition}

Let $\pi: R[V\times W]\to V \otimes_\RR W$ be the canonical projection and write $v\otimes_R w$ for
the image of $(v, w)\in R[V\times W]$. It is clear from \eqref{eq:16-1} that $\pi:V\times W\to V\otimes_R W$
is $R$-bilinear. Moreover, it is universal with this property in the following sense:

\begin{lemma}\label{lemma:16-2}
  Let $V, W$ and $U$ be $R$-modules, and let $f: V\times W\to U$ be any
  $R$-bilinear map. Then there exists a unique $R$-linear map $\overline{f}:V\otimes_R W\to U$, with
  $f = \overline{f}\circ \pi$.
\end{lemma}

\begin{proof}
  Since the set $V\times W$ is a basis for the $R$-module $R[V\times W]$, f extends
  to an $R$-linear map $\hat f: R[V\times W]\to U$. The bilinearity of $f$ implies that
  $\hat f(R(V, W)) = 0$, so that $\hat f$ induces a map $\overline{f}$ from the quotient $V\otimes_R W$ to
  $U$. By construction $f = \overline{f}\circ \pi$, $\overline{f}$ is R-linear and since $\pi(V\times W)$ 
  generates the $R$-module $V\otimes_R W$, $\overline{f}$ is uniquely determined by $f$.
\end{proof}

It is immediate from Lemma \ref{lemma:16-2} that tensor product is a functor. Indeed, if
$\varphi:V\to V'$ and $\psi:W\to W'$ are $R$-linear maps then the composition
\[
  V\times W \xra[\varphi\times\psi] V'\times W'\xra[\pi'] V'\otimes_R W'
\]
is bilinear, so induces a unique map 
\begin{align}\label{eq:16-2}
  \varphi\otimes_R\psi: V\otimes_R W\to V'\otimes_R W'
\end{align}

The uniqueness guarantees that $(\varphi'\otimes_R\psi')\circ(\varphi\otimes_R\psi) 
= \varphi'\circ\varphi\otimes_R\psi'\circ\psi$ when $\varphi':V'\to V'', \psi':W'\to W''$.

In the general setting of modules over a commutative ring, not every $R$-module
$V$ has a basis, i.e. is of the form $R[B]$ for a subset $B\subset V$, but if it does we
say that $V$ is a free $R$-module.

\begin{lemma}\label{lemma:16-3}
  Let $V$ and $V'$ befree $R$-modules with bases $B$ and $B'$. Then $V\otr V'$
  is a free $R$-module with basis $\{b\otr b'\mid b\in B, b'\in B'\}$.
\end{lemma}

\begin{proof}
  The bilinearity of $\pi:V\times V'\to V\otr V'$ shows that the stated set generates $V\otr V'$; 
  so suppose that
  \begin{align}\label{eq:16-3}
    \sum r_{ij}b_i\otr b_j' = 0
  \end{align}
  is a (finit) relation. Let $\varphi_0:V\to R$ and $\varphi_0':V'\to R$ be the linear maps with 
  \[\begin{matrix}
    \varphi_0(b_i) = 0 \text{ if } i\neq i_0, \varphi_0(b_{i_0}) = 1 \\
    \varphi_0'(b_j') = 0 \text{ if } j\neq j_0, \varphi_0'(b_{j_0}') = 1
  \end{matrix}\]
  where $(i_0, j_0)$ is a pair of indices that appear in \eqref{eq:16-3}. The composition
  \[
    V\otr V' \xra[\varphi_0\otimes\varphi_0'] R\otr R \xra*[mult] R
  \]
  maps the left-hand side in \eqref{eq:16-3} into $r_{i_0j_0}$ which must therefore be zero.

  We note the general relations
  \begin{align}\label{eq:16-4}
    \begin{aligned}
      \text{(i)}\qquad & R\otr V \simee V \simee V\otr R \\
      \text{(ii)}\qquad & V_1\otr V_2 \simee V_2\otr V_1 \\
      \text{(iii)}\qquad & V_1\otr(V_2\otr V_3)\simee (V_1\otr V_2)\otr V_3 \\
      \text{(iv)}\qquad & (V_1\otr V_2)\otr V_3 \simee V_1\otr V_2 \oplus V_1\otr V_3.
    \end{aligned}
  \end{align}

  They all follow from the universal property of Lemma \ref{lemma:16-2}. For example, scalar
  multiplication defines an $R$-bilinear map from $R\times V$ into $V$, and hence a map
  from $R\otr V$ into $V$ whose inverse is the map that sends $v\in V$ into $1\otr v$. The
  other cases are equally simple.


  There is an obvious generalization of Definition \ref{definition:16-1} to $k$-variable tensor products
  $V_1\otr\ldots\otr V_k$ and a corresponding generalization of Lemma \ref{lemma:16-2}, namely

  \begin{lemma}\label{lemma:16-4}
    For every $R$-multilinear map $f:V_1\times\ldots\times V_k\to W$ there exists a unique $R$-linear map 
    $\overline{f}:V_1\otr\ldots\otr V_k\to W$ such that the following diagram commutes.
    \[\begin{tikzcd}
    V_1\times\ldots\times V_k \arrow[rr, "f"], \arrow[dr, "\pi"] &  & W \\
    & V_1\otr\cdots\otr V_k \arrow[ur, "\overline{f}"'] & 
    \end{tikzcd}\]
  \end{lemma}
\end{proof}

It is easy to see that the $k$-variable tensor product is isomorphic to the iteration
of the two-variable ones,
\[
  (\ldots((V_1\otr V_2)\otr V_3)\otr\ldots\otr V_k) \simee V_1\otr V_2\otr\ldots\otr V_k.
\]

Let us specialize to $V_1=\ldots=V_k=V$. Consider the sub-$R$-module $A^k(V)\subseteq \bigotimes_R^k V$
generated by the set 
\[
  \{v_1\otr\ldots\otr v_k\mid v_i\in V \text{ and } v_{i_0}=v_{i_0+1} \text{ for some } i_0\}.
\]

\begin{definition}\label{definition:16-5}
  The quotient module 
  \[
    \Lambda_R^k = \bigotimes_R^k (V) / A^k(V)
  \]
  is called the \Index{exterior {$k$}-th power}.
\end{definition}

The image of $v_1\otr\ldots\otr v_k$ in $A_R^k(V)$ under the canonical projection  
\[
  \pi_1: \bigotimes_R^k (V) \to \Lambda_R^k (V)
\]
is denoted $v_1\wedge_R\ldots\wedge_R v_k$. Then for a permutation $\sigma$
\begin{align}\label{eq:16-5}
  v_1\wedge_R\ldots\wedge_R v_k
  = \sign(\sigma) v_{\sigma(1)}\wedge_R\ldots\wedge_R v_{\sigma(k)}
\end{align}
as in Lemma \ref{lemma:2-2} and Lemma \ref{lemma:2-7}, so that the composition 
\begin{align}\label{eq:16-6}
  \rho = \pi_1\circ\pi:V\times\ldots\times V \to \Lambda_R^k(V)
\end{align}
is an alternating map, i.e. an $R$-multilinear map with $\rho(v_1,\ldots,v_k)=0$ when 
$v_i=v_j$ for some $i<j$. With these definitions we get from Lemma \ref{lemma:16-4}:

\begin{lemma}\label{lemma:16-6}
  For any $R$-alternating map $\omega: V \times\ldots \times V\to W$ there is a unique $R$-linear 
  map $\overline{\omega}:\Lambda_R^k\to W$ such that $\omega = \overline{\omega}\circ\rho$.\hfill $\square$
\end{lemma}

In the special case where $R = \RR$ (or $\CC$) and where we restrict attention to finite 
dimensional vector spaces the exterior powers $\Lambda_R^k(V)$ are dual to the alternating
power $\alt^k(V)$ introduced in Chapter 2. To simplify notation we drop the subscript $\RR$ and write
$\otimes^k(V), \Lambda^k(V), V\otimes W, \hom(V, W)$ instead of $\otimes_\RR^k(V), \Lambda_\RR^k(V), 
V\otr[\RR] W, \hom_\RR(V, W)$.

Let $V^*=\hom_\RR(V, \RR)$ be the dual vector space. The exterior product introduced in Definition \ref{definition:2-5}
defines an alternating map 
\[
  \varphi:V^*\times\ldots\times V^*\to \alt^k(V)
\]
and thus by Lemma \ref{lemma:16-6} a linear map 
\[
  \overline{\varphi}:\Lambda^k V^*\to \alt^k(V).
\]

\begin{theorem}\label{theorem:16-7}
  \begin{enumerate}
    \item The map $\overline{\varphi}$ is an isomorphism.
    \item If $\{e_i\}_{i=1}^n$ is a basis for $V$ then $\{e_{i_1}\wedge\ldots\wedge e_{i_k}\mid i_1<\ldots<i_k\}$
      is a basis for $\Lambda^k V$.
    \item There is a natural isomorphism $\Lambda^k(V^*)\simee\Lambda^k(V)^*$.
  \end{enumerate}
\end{theorem}

\begin{proof}
  The map $\overline{\varphi}$ is surjective by Theorem \ref{theorem:2-15}, and hence $\dim\Lambda^k(V^*)\ge \binom{n}{k}$.
  On the other hand Lemma \ref{lemma:16-3} and \eqref{eq:16-5} imply that the set in (ii) generates $\Lambda^k V$, so that 
  $\dim\Lambda^k V\le \binom{n}{k}$. Since $\dim\Lambda^k V = \dim\alt^k(V^*)$, the common dimension is $\binom{n}{k}$.
  Thus $\overline{\varphi}$ is an isomorphism. This proves (i) and (ii). For each fixed $\omega\in\alt^k(V), \omega(v_1, \ldots, v_k)$
  is an alternating map, and so defines by Lemma \ref{lemma:16-6} a linear map from $\Lambda^k V^*$ to $\RR$. This gives a map 
  \[
    \overline{\psi}:\alt^k(V)\to \Lambda^k(V)^*
  \]
  which is linear and injective. Since the dimensions on both sides agree, $\overline{\psi}$ is an
  isomorphism, and (iii) follows from (i).
\end{proof}

Consider the bilinear pairing
\[
  \wedge:\Lambda^k(V)\times\Lambda^l(V)\to \Lambda^{k+l}(V)
\]
which takes $(v_1\wedge\ldots\wedge v_k, w_1\wedge\ldots\wedge w_l)$ 
into $v_1\wedge\ldots\wedge v_k\wedge w_1\wedge\ldots\wedge w_l$. Then 
\[
  \overline{\varphi}:\Lambda^k (V^*)\to \alt^k(V)
\]
becomes multiplicative, $\overline{\varphi}(\omega_1\wedge\omega_2)=\overline{\omega_1}\wedge\overline{\varphi}(\omega_2)$, where 
the product on the range is the one from Definition \ref{definition:2-5}. The composition
\[
  \psi:\Lambda^k(V^*)\otimes \Lambda^l(V) \xra[\overline{\varphi}\otimes 1] 
  \alt^k(V)\otimes \Lambda^k(V) \xra*[ev] \RR
\]
with $\R{ev}(\omega, v_1\wedge\ldots\wedge v_k) = \omega(v_1,\ldots,v_k)$ induces the 
isomorphism $\overline{\psi}:\Lambda^k(V^*)\simee \Lambda^k(V)^*$
in Theorem \ref{theorem:16-7}.(iii) On the other hand, Lemma \ref{lemma:2-13} gives that 
\begin{align}\label{eq:16-7}
  \psi((\epsilon_1\wedge\ldots\wedge\epsilon_{i_k})\otimes (v_1\wedge\ldots\wedge v_k))
  = \det((\epsilon_i(v_j))_{i,j=1}^k).
\end{align}
In particular, if $V$ has an inner product $\langle\, ,\,\rangle$, then we may identity $V$ with $V^*$ by 
sending $v$ to $\langle v, -\rangle$, and $\psi$ becomes the map 
\[
  \psi(w_1\wedge\ldots\wedge w_k, v_1\wedge\ldots\wedge v_k) 
  = \det\left(\langle w_i, v_j\rangle_{i,j=1}^k\right)
\]

Theorem \ref{theorem:16-7}.(iii) then translates into 
\begin{addendum}[Grassmann inner product]\label{addendum:16-8}\index{Grassmann inner product}
  For an inner product space $V$, the formula
  \[
    \langle w_1\wedge\ldots\wedge w_k, v_1\wedge\ldots\wedge v_k\rangle
    = \det\left(\langle w_i, v_j\rangle_{i,j=1}^k\right)
  \]
  defines an inner product on $\Lambda^k(V)$.\hfill $\square$
\end{addendum}

For vector spaces $V$ and $W$ there is an obvious linear map
\[
  V^*\otimes W\to \hom(V, W)
\]
which takes $f\otimes w$ into the linear map $v\ma f(v)w$. It follows from Lemma \ref{lemma:16-3}
that this is an isomorphism
\begin{align}\label{eq:16-8}
  V^*\otimes W \simee \hom(V, W);\quad \dim W< \infty.
\end{align}

The above constructions on vector spaces induce constructions on vector bundles
(over the same base space) by applying them fiberwise. So if $\xi$ and $\eta$ are (smooth)
vector bundles over $X$ then we get new (smooth) vector bundles over $X$:
\begin{align}\label{eq:16-9}
  \xi\otimes\eta, \qquad \hom(\xi, \eta), \qquad \otimes^k(\xi), \qquad \Lambda^k(\xi), 
  \qquad \xi^*, \qquad \alt^k(\xi)
\end{align}
and so forth. The fiber over $x\in X$ in each case is the associated construction
on $\xi_x, \eta_x$. To be more specific, let us run through the definition in one of the
cases, say $\hom(\xi, \eta)$. We define
\[
  E = E(\hom(\xi, \eta)) = \coprod_{x\in X} \hom(\xi_x, \eta_x)
\]
with the obvious projection map $\pi$ onto $X$, which maps the entire vector space
$\hom(\xi_x, \eta_x)$ to $x\in X$. The problem is to define a topology on $E$ that makes
this the total space of a vector bundle.

Let $\{U_j\}_{j\in J}$ be an open cover of $X$ for which both $\xi|_{U_j}$ and $\eta|_{U_j}$ are trivial.
Choose isomorphisms
\[
  h_j:\xi_{|U_j}\simee U_j\times\RR^{n}, \qquad
  k_j:\eta_{|U_j}\simee U_j\times\RR^{m}
\] 
They induce inclusions
\[
  U_j\times \hom(\RR^n, \RR^m)\xra[H_j] E(\hom(\xi, \eta))
\]
(which depend on $h_j$ and $k_j$). We can use $\{h_j\}$ to define a topology on $E$:
\begin{align}\label{eq:16-10}
  A\subset E \text{ open } \Leftrightarrow H_j^{-1}(A) \text{ open for all } j.
\end{align}
where the topology on $U_j\times \hom(\RR^n, \RR^m)$ is the product topology. It is left
to the reader to show that \eqref{eq:16-10} defines a topology and that it is independent of
the choice of cover and isomorphisms $h_j, k_j$. This uses that $\hom(V, W)$ is a
continuous functor, i.e. that the maps
\[
  \begin{aligned}
    & T:\hom(V, V_1)\times \hom(W_1, W)\to \hom(\hom(V_1, W_1), \hom(V, W));\\[.25em]
    & T(f, g)(\phi) = g\circ\phi\circ f
  \end{aligned}
\]
are continuous. In particular note that $\xi\simee X\times V$ and $\eta\simee X\times W$ give $\hom(\xi, \eta)\simee X\times \hom(V, W)$.

All of constructions in \eqref{eq:16-9} produce smooth vector bundles when $\xi$ and $\eta$ are smooth.

\begin{lemma}\label{lemma:16-9}
  For (smooth) vector bundles $\xi$ and $\eta$ there are isomorphisms $\xi\simee \xi^{**}$ and $\xi^*\otimes \eta\simee\hom(\xi, \eta)$.
\end{lemma}

\begin{proof}
  For finite-dimensional vector spaces there are natural isomorphisms 
  \[
    V\to V^{**}, \qquad V^*\otimes W\to\hom(V, W)
  \]
  defined without any reference to basis. This gives maps of vector bundles (over the identity)
  \[
    \xi\to \xi^{**}, \qquad \xi^*\otimes\eta\to\hom(\xi, \eta)
  \]
  which are isomorphisms on each fiber, and we can apply Lemma \ref{lemma:15-10}.
\end{proof}

For finite-dimensional vector spaces $V\simee V*$, since the dimensions agree. However, 
it is not true in general that $\xi\simee\xi^*$; cf. Properties \ref{property:18-11} below. The above
proof breaks down because there is no isomorphism from $V$ to $V^*$ defined independently of choice of basis. 
As a result one cannot define a homomorphism from $\xi$ to $\xi^*$.

The isomorphisms of Theorem \ref{theorem:16-7}.(iii) are again defined without reference to a
basis, and so give isomorphisms
\begin{align}\label{eq:16-11}
  \Lambda^k(\xi^*)\simee \alt^k(\xi)
\end{align}
for all vector bundles $\xi$.

We have in Theorem \ref{theorem:16-7}, Addendum \ref{addendum:16-8}, Lemma \ref{lemma:16-9}, and \eqref{eq:16-11} 
concentrated on real vector spaces and vector bundles, and leave the reader to formulate and
prove the corresponding results for the complex cases.


Every complex vector bundle $\xi$ gives rise to a real vector bundle $\xi_\RR$ upon forgetting
part of the structure. On the other hand, every real vector bundle induces a
complex vector bundle $\eta_\CC$ upon complexifying each fiber, $\eta_\CC=\eta\otr[\RR] \varepsilon_\CC^1$ 
where $\varepsilon_\CC^1$ denotes the trivial complex line bundle. There are the following relations
between these operations.

\begin{lemma}\label{lemma:16-10}
  \begin{enumerate}
    \item For a real vector bundle $\eta$, $(\eta_\CC)_\RR\simee \eta\oplus\eta$.
    \item For a complex vector bundle $\xi$, $(\xi_\RR)_\CC\simee \xi\oplus\xi^*$.
  \end{enumerate}
\end{lemma}

\begin{proof}
  We leave (i) to the reader and prove (ii). Given a complex vector space $V$ we get a new 
  one $V\otr[\RR]\CC$ where scalar multiplication is in the second factor. The map
  \[
    \varphi:V\otr[\RR]\CC\to V\oplus V;\qquad 
    \varphi(v\otimes z) = (zv, \overline{z}v)
  \]
  defines an isomorphism of the underlying real vector spaces, and $\varphi$ becomes
  complex linear if, in the second summand of the target, we replace $V$ by its
  conjugate vector space $\overline{V}$, in which multiplication by $z\in\CC$ is replaced by
  multiplication with the complex conugate $\overline{z}$. Suppose that $V$ has a hermitian
  product. Then we may identify $\overline{V}$ with the complex dual vector space $V^*$ via the
  map $v\ma \langle -, v\rangle$, and $\varphi$ gives a complex linear isomorphism
  \[
    \varphi:V\otr[\RR]\CC \to V\oplus V^*.
  \]
  We choose a hermitian product on $\xi$, and apply $\varphi$ fiberwise to prove (ii).
\end{proof}

We next consider spaces of sections. We shall primarily be interested in the space
of smooth sections $\Omega^0(\xi)$ of smooth vector bundles; cf. Definition \ref{definition:15-15}. 
For the tangent bundle $\tau_M$,
\begin{align}\label{eq:16-12}
  \Omega^0(\alt^k(\tau_M)) = \Omega^k(M).
\end{align}
Indeed, an element of the left-hand side associates to each $x\in M$ an element $\omega_x\in \alt^k(T_xM)$, 
and sine $\Omega^0(-)$ denotes the space of smooth sections, $\{\omega_x\}_{x\in M}$ is a smooth $k$-form in nthe sense of 
Definition \ref{definition:9-15}. Similarly $\Omega^0(\tau_M)$ is the space of smooth tangent vector fields on $M$.

The section space $\Omega^0(\xi)$ of a smooth vector $M$ is always a module over the ring $\Omega^0(M)=C^\infty(M, \RR)$;
if $s\in\Omega^0(\xi)$ and $f\colon M\to\RR$ is a smooth map then $(fs)(x)=f(x)s(x), x\in M$ is a new smooth section of $\xi$. We can 
apply Definition \ref{definition:16-1} and \ref{definition:16-5} with $R = \Omega^0(M)$ and $V, W=\Omega^0(\xi), \Omega^0(\eta)$, eta.

The $\Omega^0(M)$-module $\Omega^0(\xi)$ is not in general a free module. Indeed, if it is then a choice of $\Omega^0(M)$-basis
$e_1, \ldots, e_k\in \Omega^0(\xi)$ has the property that $e_1(p), \ldots, e_k(p)$ is a basis for $\xi_p$ for every $p\in M$ and $\xi$
must be trivial.


\begin{lemma}\label{lemma:16-11}
  For every smooth vector bundle $\xi$ over a compact smooth manifold $M, \Omega^0(\xi)$ is 
  a direct $\Omega^0(M)$-summand in afinitely generatedfree $\Omega^0(M)$-module.
\end{lemma}


\begin{proof}
  By Theorem \ref{theorem:15-18} there is a complement $\eta$ to $\xi$, $\xi\oplus\eta\simee \varepsilon^{k+l}$. Then 
  \[
    \Omega^0(\xi)\oplus\Omega^0(\eta) \simee 
    \Omega^0(\xi\oplus\eta)\simee 
    \Omega^0(\varepsilon^{k+l})
  \]
  and $\Omega^0(\varepsilon^{k+l})$ is a free $\Omega^0(M)$-module (of dimension $k+l$).
\end{proof}

Direct summands in free modules are called projective modules\index{projective module}, so the above
lemma tells us that $\Omega^0(\xi)$ is always a finitely generated projective $R$-module,
with $R = \Omega^0(M)$.

\begin{lemma}\label{lemma:16-12}
  Let $P_1$ and $P_2$ be finitely generated projective $R$-modules. Then there are isomorphisms
  \[
    P_1\simee P_1^{**}, \qquad \hom_R(P_1, P_2)\simee P_1^*\otr P_2
  \]
  where $P_1^*=\hom_R(P_1, R)$.
\end{lemma}

\begin{proof}
  One first proves the assertions for finitely generated free modules, where
  the argument is completely similar to the case of vector spaces. The general case
  follows easily upon choosing complements $P_1\oplus Q_1=R^{n_1}, P_2\oplus Q_2=R^{n_2}$.
  Details are left as an exercise.
\end{proof}

\begin{theorem}\label{theorem:16-13}
  There are the following isomorphisms
  \begin{enumerate}
    \item $\Omega^0(\hom(\xi, \eta))\simee \hom_{\Omega^0(M)}(\Omega^0(\xi), \Omega^0(\eta))$.
    \item $\Omega^0(\xi\otimes\eta) \simee \Omega^0(\xi)\otimes_{\Omega^0(M)}\Omega^0(\eta)$.
    \item $\Omega^0(\xi^*)\simee \hom_{\Omega^0(M)}(\Omega^0(\xi), \Omega^0(M))$.
    \item $\Omega^0(\Lambda^i\xi) \simee \Lambda^i_{\Omega^0(M)}(\Omega^0(\xi))$.
  \end{enumerate}
\end{theorem}

\begin{proof}
  By definition, $\Omega^0(\hom(\xi, \eta))$ is the space of smooth fiberwise homomorphisms $\hat{\varphi}:\xi\to\eta$,
  and we define an $\Omega^0(M)$-linear homomorphism
  \[
    F:\Omega^0(\hom(\xi, \eta))\to \hom_{\Omega^0(M)}(\Omega^0(\xi), \Omega^0(\eta))
  \]
  by $F(\hat{\varphi})(s)=\hat{\varphi}\circ s, s\in \Omega^0(\xi)$. Suppose $F(\hat\varphi) = 0$. 
  To check injectivity of $F$ we must show that $\hat\varphi_x:\xi_x\to\eta_x$ is the zero homomorphism for every $x\in M$.
  Fix $x\in M$ and $v\in\xi_x$. There is a section $s_v\in\Omega^0(\xi)$ with $s_v(x) = v$. Indeed,
  such an $s_v$ can always be defined in a neighborhood $U$ of $x\in M$ where $\xi_{|U}$ is
  trivial, and we can obtain a global section upon choosing a smooth function $f$ on $M$ with $\supp(f)\subset U$ 
  and $f(x) = 1$ (cf. Appendix A) and replace the local section $s_v$ by $fs_v$, giving it the value zero outside $U$.

  Now $F(\hat\varphi)(s_v)=0$ implies that $\hat\varphi(v)=0$. Since $x$ and $v$ were arbitrary. $F$ is injective.

  Let $\Phi\in\hom_{\Omega^0(M)}(\Omega^0(\xi), \Omega^0(\eta))$. We wish todefine a fiberwise smooth homomorphism 
  $\hat\varphi:\xi\to \eta$ by setting $\hat\varphi(v)=\Phi(s_v)(x)$ where $s_v\in\Omega^0(\xi)$ is a section with $s_v(x)=v$. 
  This requires that one has
  \[
    \Phi(s_v)(x) = \Phi(s_v')(x)
  \]
  for two sections of $\xi$ with $s_v(x)=s_v'(x)$, or in other words that 
  \begin{align}\label{eq:16-13}
    \text{if}\quad s(x) = 0 \text{ then } \Phi(s)(x) = 0.
  \end{align}
  Choose sections $e_1, \ldots, e_k\in\Omega^0(\xi)$ such that $e_1(p), \ldots, e_k(p)$ form a basis for 
  $\xi_p$ for $p$ in some neighborhood $U$ of $x$. Again this can be done locally, and the local sections can be extended
  to global ones as above. Now
  \[
    s(p) = \sum f_i(p)e_i(p)\qquad \text{for } p\in U
  \]
  for smooth functions $f_i$ defined on $U$. Let $\lambda\in\Omega^0(M)$ have $\R{Supp}(\lambda)\subset U$ 
  and $\lambda(x)=1$. Then 
  \[
    \Phi(s) = \Phi(\lambda s + (1-\lambda)(s))
    = \Phi(\lambda s) + (1-\lambda)\Phi(s)
  \]
  so that $\Phi(s)(x) = \Phi(\lambda s)(x)$. But $\lambda s=\sum (\lambda f_i)e_i$ and $\lambda f_i$ extends to a smooth 
  function $g_i$ defined on all of $M$ with $g_i(x) = f_i(x)=0$. Since $\Phi$ is $\Omega^0(M)$-linear,
  \[
    \Phi(\lambda s) = \sum g_i\Phi(e_i)\in \Omega^0(\eta).
  \]
  But $g_i(x)=0$ so $\Phi(\lambda s)(x) = 0$. This proves (i).

  assertion (iii) is the special case of (i) corresponding to $\eta=\varepsilon^1$, the trivial line bundle, 
  and (ii) follows from Lemma \ref{lemma:16-9} and (i):
  \begin{align*}
    \Omega^0(\xi\otimes\eta)
    & \simee \Omega^0(\hom(\xi^*, \eta)) 
      \simee \hom_{\Omega^0(M)}(\Omega^0(\xi^*), \Omega^0(\eta)) \\
    & \simee \hom_{\Omega^0(M)}(\hom_{\Omega^0(M)}(\Omega^0(\xi), \Omega^0(M)), \Omega^0(\eta))\\
    & \simee \Omega^0(\xi)\otimes_{\Omega^0(M)}\Omega^0(\eta)
  \end{align*}
  where the last isomorphism is from Lemma \ref{lemma:16-11} and Lemma \ref{lemma:16-12}. Finally there 
  is a commutative diagram
  \[\begin{tikzcd}
    \Omega^0(\bigotimes^i\xi)\dar{}\rar{} & \bigotimes_{\Omega^0(M)}(\Omega^0(\xi))\dar{} \\
    \Omega^0(\Lambda^i\xi)\rar{} & \Lambda^i_{\Omega^0(M)}(\Omega^0(\xi))
  \end{tikzcd}\]
  By (ii), the upper horizontal map is an isomorphism. To see that the bottom
  homomorphism is also an isomorphism, one can use Theorem \ref{theorem:16-7}.(ii), and local
  sections as in the proof of (i). The details are left as an exercise.
\end{proof}

We close with a weaker form of the universal property of tensor products, stated
in Lemma \ref{lemma:16-2}. Let $R$ be a unital commutative ring and $S$ an $R$-algebra. In our
applications in the next chapter $R = \RR$ and $S = \Omega^0(M)$, the smooth functions
on $M$, or their complex versions $R = \CC$ and $S = \Omega^0(M; \CC)$. Suppose that $V$
and $W$ are $R$-modules and that $S$ operates from the right on $V$ and from the
left on $W$, e.g.
\[
  V = \Omega^i(M), \qquad W = \Omega^0(\xi).
\]

\begin{definition}\label{definition:16-14}
  The \Index{balanced product} (or tensor product) $V\otr[S] W$ is the cokenel of the $R$-linear homomorphism
  \[
    V\otr S\otr W \xra[\beta] V\otr W
  \]
  given by $\beta(v\otr s\otr w) = vs\otr w - v\otr sw$.
\end{definition}

When $S$ is commutative, and this will be the case in our applications, then there
are no distinctions between left and right actions of $S$, and $V\otr[S] W$ becomes an
$S$-module upon defining $s(v\otr[S] w) = vs\otr[S] w$.


This $S$-module is obviously isomorphic to the one defined in Definition \ref{definition:16-1}. We
record for later use the obvious
\begin{lemma}\label{lemma:16-15}
  Let $f:V\otr W\to U$ be an $R$-linear map which is $S$-balanced in the sense that $f(vs\otr w)=f(v\otr sw)$ 
  for $f\in V, w\in W$ and $s\in S$. Then there is an induced $R$-linear map $\overline{f}:V\otr[S] W\to U$.
  \hfill $\square$
\end{lemma}